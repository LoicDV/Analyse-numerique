\documentclass[a4paper, 10pt]{article}
\usepackage[utf8]{inputenc}
\usepackage[pdftex]{graphicx}
\usepackage{tabularx}
\usepackage[french, english]{babel}
\usepackage{geometry}
\usepackage{booktabs}
\usepackage{hyperref}
\usepackage{verbatim}
\usepackage{pdfpages}
\geometry{hmargin=2cm,vmargin=1.5cm}
%\renewcommand{\baselinestretch}{0.96}
\newcommand{\IR}{\mathbb{R}}

\begin{document}

\begin{titlepage}
\begin{center}

{\Large Université de Mons}\\[1ex]
{\Large Faculté des Sciences}\\[1ex]
{\Large Département des mathématiques}\\[2.5cm]

\newcommand{\HRule}{\rule{\linewidth}{0.3mm}}
% Title
\HRule \\[0.3cm]
{ \LARGE \bfseries Rapport du TP2 \\[0.3cm]}
{ \LARGE \bfseries Analyse Numérique \\[0.1cm]}
\HRule \\[1.5cm]

% Author and supervisor
\begin{minipage}[t]{0.45\textwidth}
\begin{flushleft} \large
\emph{Professeur:}\\
Christophe \textsc{Troestler} \\
Quentin \textsc{Lambotte}
\end{flushleft}
\end{minipage}
\begin{minipage}[t]{0.45\textwidth}
\begin{flushright} \large
\emph{Auteurs:} \\
Loïc \textsc{Dupont} \\
Paolo \textsc{Marcelis} \\
Maximilien \textsc{Vanhaverbeke}
\end{flushright}
\end{minipage}\\[2ex]

\vfill

% Bottom of the page
\begin{center}
\begin{tabular}[t]{c c c}
\includegraphics[height=1.5cm]{logoumons.jpg} &
\hspace{0.3cm} &
\includegraphics[height=1.5cm]{logofs.jpg}
\end{tabular}
\end{center}~\\
 
{\large Année académique 2019-2020}

\end{center}
\end{titlepage}

\renewcommand{\contentsname}{Table des matières}
\tableofcontents

\newpage

Soit \(A \in \IR^{M \times N} \text{ avec rang } A = N \leq M \). \\
Intéressons-nous à la résolution de l'équation \( Ax = b \). \\
Les exercices suivants consistent à regarder la meilleur solution au sens des moindres carrés. \\
C'est-à-dire, la valeur de x qui minimise la fonction :
\[
	x \mapsto ~ \mid Ax - b \mid_2 \label{eq1: norme deux}
\]
\section{Exercice 1 :}


















\end{document}