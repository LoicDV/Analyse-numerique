\documentclass[a4paper, 10pt]{article}
\usepackage[utf8]{inputenc}
\usepackage[pdftex]{graphicx}
\usepackage{tabularx}
\usepackage{amsmath}
\usepackage{amssymb}
\usepackage{geometry}
\usepackage{booktabs}
\usepackage{hyperref}
\usepackage{verbatim}
\usepackage{pdfpages}
\geometry{hmargin=2cm,vmargin=1.5cm}
%\renewcommand{\baselinestretch}{0.96}
\newcommand{\IR}{\mathbb{R}}
\newcommand{\IN}{\mathbb{N}}
\newcommand{\inte}{\operatorname{int}}

\begin{document}

\begin{titlepage}
\begin{center}

{\Large Université de Mons}\\[1ex]
{\Large Faculté des Sciences}\\[1ex]
{\Large Département des mathématiques}\\[2.5cm]

\newcommand{\HRule}{\rule{\linewidth}{0.3mm}}
% Title
\HRule \\[0.3cm]
{ \LARGE \bfseries Rapport du TP2 \\[0.3cm]}
{ \LARGE \bfseries Analyse Numérique \\[0.1cm]}
\HRule \\[1.5cm]

% Author and supervisor
\begin{minipage}[t]{0.45\textwidth}
\begin{flushleft} \large
\emph{Professeur:}\\
Christophe \textsc{Troestler} \\
Quentin \textsc{Lambotte}
\end{flushleft}
\end{minipage}
\begin{minipage}[t]{0.45\textwidth}
\begin{flushright} \large
\emph{Auteurs:} \\
Loïc \textsc{Dupont} \\
Paolo \textsc{Marcelis} \\
Maximilien \textsc{Vanhaverbeke}
\end{flushright}
\end{minipage}\\[2ex]

\vfill

% Bottom of the page
\begin{center}
\begin{tabular}[t]{c c c}
\includegraphics[height=1.5cm]{logoumons.jpg} &
\hspace{0.3cm} &
\includegraphics[height=1.5cm]{logofs.jpg}
\end{tabular}
\end{center}~\\
 
{\large Année académique 2020-2021}

\end{center}
\end{titlepage}

\renewcommand{\contentsname}{Table des matières}
\tableofcontents

\newpage

\noindent
Soit \( M, N \in \IN \). Soit \( A \in \IR^{M \times N} \) avec rang\( A = N \leq M \). Soit \( b \in \mathbb{R}^n \). \\
Intéressons-nous à la résolution de l'équation \( Ax = b \). \\
Les exercices suivants consistent à regarder la meilleur solution au sens des moindres carrés càd la valeur de x qui minimise la fonction :
\begin{equation}
    \label{eq:1}
    x \mapsto |Ax - b|_2
\end{equation}
Pour les questions suivantes, nous allons utiliser les notations ci-dessous :
$$
A =
\begin{pmatrix}
a_{1, 1} & \cdots & a_{1, N} \\
\vdots   &        & \vdots \\
a_{M, 1} & \cdots & a_{M, N}
\end{pmatrix}
= (a_{i, j})_{
\tiny \begin{matrix}
1 \leq i \leq M \\
1 \leq j \leq N
\end{matrix}}
, \quad b =
\begin{pmatrix}
b_1 \\
\vdots \\
b_M
\end{pmatrix}
= (b_i)_{1 \leq i \leq M}, \quad x =
\begin{pmatrix} x_1 \\
\vdots \\
x_N
\end{pmatrix}
= (x_j)_{1 \leq j \leq N}
$$

\section{Exercice 1}

Soit \( x \in \IR^N \). On doit montrer que \( x \) réalise le minimum de \( x \mapsto |Ax - b|_2 \) \eqref{eq:1} si et seulement si \( x \) réalise le minimum de
\begin{equation}
    \label{eq:2}
    x \mapsto |Ax - b|_2^2
\end{equation}
Commençons par montrer que si \( x \) réalise le minimum de \eqref{eq:1}, alors \( x \) réalise le minimum de \eqref{eq:2}. \\
Supposons que \( x \) réalise le minimum de \eqref{eq:1} càd
$$
\forall y \in \IR^N,~ |Ax - b|_2 \leq |Ay - b|_2
$$
Comme toute norme est définie positive, on sait que pour tout \( z \in \IR^M,~ |z|_2 \in [0, +\infty[ \). \\
Alors, comme chaque membre de l'inégalité ci-dessus est positif, par croissance de la fonction \( x \mapsto x^2 \) sur \( [0, +\infty[ \), on a
$$
\forall y \in \IR^N,~ |Ax - b|_2^2 \leq |Ay - b|_2^2
$$
càd \( x \) réalise le minimum de \eqref{eq:2}. \\
Il reste à montrer que si \( x \) réalise le minimum de \eqref{eq:2}, alors \( x \) réalise le minimum de \eqref{eq:1}. \\
Supposons que \( x \) réalise le minimum de \eqref{eq:2} càd
$$
\forall y \in \IR^N,~ |Ax - b|_2^2 \leq |Ay - b|_2^2
$$
Comme tout carré d'un nombre réel est positif, par croissance de la fonction \( x \mapsto \sqrt{x} \) sur \( [0, +\infty[ \), on a
$$
\forall y \in \IR^N,~ \sqrt{|Ax - b|_2^2} \leq \sqrt{|Ay - b|_2^2}
$$
Comme les normes sont définies positives, \( \sqrt{|Ax - b|_2^2} = |Ax - b|_2 \) et \( \forall y \in \IR^N,~ \sqrt{|Ay - b|_2^2} = |Ay - b|_2 \). \\
Par conséquent, on a bien
$$
\forall y \in \IR^N,~ |Ax - b|_2 \leq |Ay - b|_2
$$
càd \( x \) réalise le minimum de \eqref{eq:1}. \\
On a donc bien montré que \( x \) réalise le minimum de \eqref{eq:1}, alors \( x \) réalise le minimum de \eqref{eq:2}.

\newpage

\section{Exercice 2}

Soit \( x \in \IR^N \). On doit montrer que \( x \) réalise le minimum de \( x \mapsto |Ax - b|_2 \) \eqref{eq:1} si et seulement si \( x \) vérifie l'équation
\begin{equation}
    \label{eq:3}
    A^T A x = A^T b
\end{equation}
Commençons par montrer que si \( x \) réalise le minimum de \eqref{eq:1}, alors \( x \) vérifie l'équation \eqref{eq:3}. \\
Supposons que \( x \) réalise le minimum de \eqref{eq:1} càd
$$
\forall y \in \IR^N,~ |Ax - b|_2 \leq |Ay - b|_2
$$
Par l'exercice 1, on sait que c'est équivalent à dire que
$$
\forall y \in \IR^N,~ |Ax - b|_2^2 \leq |Ay - b|_2^2
$$
Commençons par calculer explicitement \( f : x \mapsto |Ax - b|_2^2 \). On a
\begin{align*}
    |Ax - b|_2^2
    & = \left|
        \begin{pmatrix}
        a_{1, 1} & \cdots & a_{1, N} \\
        \vdots   &        & \vdots \\
        a_{M, 1} & \cdots & a_{M, N}
        \end{pmatrix}
        \begin{pmatrix}
        x_1 \\
        \vdots \\
        x_N
        \end{pmatrix}
        -
        \begin{pmatrix}
        b_1 \\
        \vdots \\
        b_M
        \end{pmatrix}
        \right|_2^2 \\
    & = \left|
        \begin{pmatrix}
        \displaystyle \left( \sum_{j = 1}^N a_{1, j} x_j \right) - b_1 \\
        \vdots \\
        \displaystyle \left( \sum_{j = 1}^N a_{M, j} x_j \right) - b_M
        \end{pmatrix}
        \right|_2^2 \\
    & = \left( \sqrt{\sum_{i = 1}^M \left( \left( \sum_{j = 1}^N a_{i, j} x_j \right) - b_i \right)^2} \right)^2 \\
    & = \sum_{i = 1}^M \left( \left( \sum_{j = 1}^N a_{i, j} x_j \right) - b_i \right)^2
\end{align*}
On sait donc que \( x \) réalise le minimum de
$$
f : \IR^n \to \IR : x = (x_1, \cdots, x_n) \mapsto \sum_{i = 1}^M \left( \left( \sum_{j = 1}^N a_{i, j} x_j \right) - b_i \right)^2
$$
En particulier, il réalise également le minimum de toute restriction de \( f \) à un ensemble contenant \( x \). \\
On sait donc que pour tout \( k \in \IN^{\leq n} \), \( x \) réalise le minimum de
$$
f_k : \IR \to \IR : x_k \mapsto \sum_{i = 1}^M \left( \left( \sum_{j = 1}^N a_{i, j} x_j \right) - b_i \right)^2
$$
Ces fonctions sont dérivables sur \( \IR \) et pour tout \( k \in \IN^{\leq N} \),
\begin{align*}
    \partial_k \sum_{i = 1}^M \left( \left( \sum_{j = 1}^N a_{i, j} x_j \right) - b_i \right)^2
    & = \sum_{i = 1}^M \left( \left( 2 \sum_{j = 1}^N a_{i, j} x_j \right) - 2 b_i \right) \partial_k \left( \left( \sum_{j = 1}^N a_{i, j} x_j \right) - b_i \right) \\
    & = \sum_{i = 1}^M \left( \left( 2 \sum_{j = 1}^N a_{i, j} x_j \right) - 2 b_i \right) a_{i, k}
\end{align*}

\newpage

\noindent
On sait que \( x \in \inte \IR = \IR \) et qu'il s'agit d'un minimum global de \( f_k \) et donc en particulier un minimum local. La propriété VI.16 du syllabus d'analyse 1 nous dit alors que la dérivée de \( f_k \) s'annule en \( x \). On sait donc que
\begin{align*}
    \forall k \in \IN^{\leq N},~ \partial f_k = 0
    & \quad \text{ càd } \quad \forall k \in \IN^{\leq N},~ \sum_{i = 1}^M \left( \left( 2 \sum_{j = 1}^N a_{i, j} x_j \right) - 2 b_i \right) a_{i, k} = 0 \\
    & \quad \text{ càd } \quad
        \begin{pmatrix}
        \displaystyle \sum_{i = 1}^M \left( \left( 2 \sum_{j = 1}^N a_{i, j} x_j \right) - 2 b_i \right) a_{i, 1} \\
        \vdots \\
        \displaystyle \sum_{i = 1}^M \left( \left( 2 \sum_{j = 1}^N a_{i, j} x_j \right) - 2 b_i \right) a_{i, N}
        \end{pmatrix}
        =
        \begin{pmatrix}
        0 \\
        \vdots \\
        0
        \end{pmatrix} \\
    & \quad \text{ càd } \quad
        2 \begin{pmatrix}
        \displaystyle \sum_{i = 1}^M \left( a_{i, 1} \sum_{j = 1}^N \left( a_{i, j} x_j \right) \right) \\
        \vdots \\
        \displaystyle \sum_{i = 1}^M \left( a_{i, N} \sum_{j = 1}^N \left( a_{i, j} x_j \right) \right)
        \end{pmatrix}
        -
        2 \begin{pmatrix}
        \displaystyle \sum_{i = 1}^M \left( a_{i, 1} b_i \right) \\
        \vdots \\
        \displaystyle \sum_{i = 1}^M \left( a_{i, N} b_i \right)
        \end{pmatrix}
        =
        \begin{pmatrix}
        0 \\
        \vdots \\
        0
        \end{pmatrix} \\
    & \quad \text{ càd } \quad 2 A^T A x - 2 A^T b = 0 \\
    & \quad \text{ càd } \quad A^T A x = A^T b
\end{align*}
On a donc bien montré que si \( x \) réalise le minimum de \eqref{eq:1}, alors \( x \) vérifie l'équation \eqref{eq:3}.

\end{document}