\documentclass[a4paper, 12pt]{article}
\usepackage[utf8]{inputenc}
\usepackage[pdftex]{graphicx}
\usepackage{tabularx}
\usepackage[french, english]{babel}
\usepackage{geometry}
\usepackage{pdfpages}
\usepackage{amsmath}
\usepackage{amssymb}
\usepackage{hyperref}
\usepackage{algorithmic, algorithm}

\floatname{algorithm}{Algorithme}
\newcommand{\EFFECT}{\textbf{Effet : }}
\renewcommand{\algorithmicrequire}{\textbf{Entrée :}}
\renewcommand{\algorithmicensure}{\textbf{Sortie :}}
\renewcommand{\algorithmicend}{\textbf{fin}}
\renewcommand{\algorithmicif}{\textbf{si}}
\renewcommand{\algorithmicthen}{\textbf{alors}}
\renewcommand{\algorithmicelse}{\textbf{sinon}}
\renewcommand{\algorithmicelsif}{\algorithmicelse\ \algorithmicif}
\renewcommand{\algorithmicendif}{\algorithmicend\ \algorithmicif}
\renewcommand{\algorithmicfor}{\textbf{pour}}
\renewcommand{\algorithmicforall}{\textbf{pour tout}}
\renewcommand{\algorithmicdo}{\textbf{faire}}
\renewcommand{\algorithmicendfor}{\algorithmicend\ \algorithmicfor}
\renewcommand{\algorithmicwhile}{\textbf{tant que}}
\renewcommand{\algorithmicendwhile}{\algorithmicend\ \algorithmicwhile}
\renewcommand{\algorithmicloop}{\textbf{boucle}}
\renewcommand{\algorithmicendloop}{\algorithmicend\ \algorithmicloop}
\renewcommand{\algorithmicrepeat}{\textbf{répéter}}
\renewcommand{\algorithmicuntil}{\textbf{jusqu'à}}
\renewcommand{\algorithmicprint}{\textbf{imprimer}}
\renewcommand{\algorithmicreturn}{\textbf{retourner}}
\renewcommand{\algorithmictrue}{\textbf{vrai}}
\renewcommand{\algorithmicfalse}{\textbf{faux}}

\geometry{hmargin=2cm,vmargin=1.5cm}
%\renewcommand{\baselinestretch}{0.96}

\begin{document}
\begin{titlepage}
\begin{center}

{\Large Université de Mons}\\[1ex]
{\Large Faculté des Sciences}\\[1ex]
{\Large Département des mathématiques}\\[2.5cm]

\newcommand{\HRule}{\rule{\linewidth}{0.3mm}}
% Title
\HRule \\[0.3cm]
{ \LARGE \bfseries Analyse Numérique : Rapport \\[0.3cm]}
{ \LARGE \bfseries TP 1 \\[0.1cm]}
\HRule \\[1.5cm]

% Author and supervisor
\begin{minipage}[t]{0.45\textwidth}
\begin{flushleft} \large
\emph{Professeurs:}\\
Christophe \textsc{Troestler} \\
Quentin \textsc{Lambotte}
\end{flushleft}
\end{minipage}
\begin{minipage}[t]{0.45\textwidth}
\begin{flushright} \large
\emph{Auteurs:} \\
Loïc \textsc{Dupont} \\
Maximilien \textsc{Vanhaverbeke} \\
Paulo \textsc{Marcelis}
\end{flushright}
\end{minipage}\\[2ex]

\vfill

% Bottom of the page
\begin{center}
\begin{tabular}[t]{c c c}
\includegraphics[height=1.5cm]{logoumons.jpg} &
\hspace{0.3cm} &
\includegraphics[height=1.5cm]{logofs.jpg}
\end{tabular}
\end{center}~\\
 
{\large Année académique 2020-2021}

\end{center}
\end{titlepage}

\newpage
\tableofcontents

\newpage
\section{Exercice 1:}

Soit \( I \) un intervalle. Soit \( f \in \mathcal{C}^2(I, \mathbb{R}) \). Supposons que \( \inf\{|\partial f(x)| | x \in I\} > 0 \). On doit montrer que
$$
\frac{\sup \{ | \partial^2 f(x) || x \in I \}}{2 \inf \{ |\partial f(x)|| x \in I \}}
= \sup \left\{ \left. \left| \frac{\partial^2f(\xi)}{2\partial f(\eta)} \right| \right| \xi, \eta \in I \text{ et } \partial f(\eta) \neq 0 \right\}
$$
càd
\begin{equation}
    \label{eq:1.maj}
    \frac{\sup \{ | \partial^2 f(x) || x \in I \}}{2 \inf \{ |\partial f(x)|| x \in I \}} \text{ est un majorant de }
    \left\{ \left. \left| \frac{\partial^2f(\xi)}{2\partial f(\eta)} \right| \right| \xi, \eta \in I \text{ et } \partial f(\eta) \neq 0 \right\}
\end{equation}
et
\begin{equation}
    \label{eq:1.conv}
    \exists (z_n) \subset
    \left\{ \left. \left| \frac{\partial^2f(\xi)}{2\partial f(\eta)} \right| \right| \xi, \eta \in I \text{ et } \partial f(\eta) \neq 0 \right\}
    \text{ tq } z_n \to
    \frac{\sup \{ | \partial^2 f(x) || x \in I \}}{2 \inf \{ |\partial f(x)|| x \in I \}}
\end{equation}

\subsection{Preuve de la majoration}

% \eqref{eq:1.maj} Juste pour ne pas perdre la commande, à supprimer
Comme un supremum est en particulier un majorant, on sait que
\begin{equation}
    \label{eq:1.sup}
    \forall \xi \in I,~ \sup \{ | \partial^2 f(x) || x \in I \} \geq |\partial^2 f(\xi)|
\end{equation}
De même, un infimum étant en particulier un minorant, on sait que
$$
\forall \eta \in I \text{ tq } \partial f(\eta) \neq 0,~ \inf \{ |\partial f(x)|| x \in I \} \leq |\partial f(\eta)|
$$
Comme \( \inf \{ |\partial f(x)|| x \in I \} > 0 \) par hypothèse, \( \inf \{ |\partial f(x)|| x \in I \} \in \mathbb{R}^{>0} \). On sait aussi que 
\(\forall \eta \in I \text{ tq } \partial f(\eta) \neq 0,~ |\partial f(\eta)| \in \mathbb{R}^{>0} \).
Donc, par décroissance de la fonction \( x \mapsto 1/x \) sur \( \mathbb{R}^{>0} \), on a
\begin{equation}
    \label{eq:1.inf}
    \forall \eta \in I \text{ tq } \partial f(\eta) \neq 0,~ \frac{1}{\inf \{ |\partial f(x)|| x \in I \}} \geq \frac{1}{|\partial f(\eta)|}
\end{equation}
Finalement, par \eqref{eq:1.sup} et \eqref{eq:1.inf}, on a
$$
\forall \xi, \eta \in I \text{ tq } \partial f(\eta) \neq 0,~ \frac{\sup \{ | \partial^2 f(x) || x \in I \}}{2 \inf \{ |\partial f(x)|| x \in I \}}
\geq \frac{| \partial^2f(\xi)|}{2|\partial f(\eta)|}
= \left| \frac{\partial^2f(\xi)}{2\partial f(\eta)} \right|
$$
On a donc bien montré la propriété \eqref{eq:1.maj}.

\subsection{Preuve de la suite convergente}

Par les propriétés du supremum, on sait que
$$
\exists (x_n) \subset \{ | \partial^2 f(x) || x \in I \} \text{ tq } x_n \to \sup \{ | \partial^2 f(x) || x \in I \}
$$
De même, par les propriétés de l'infimum, on sait que
$$
\exists (y_n) \subset \{ |\partial f(x)|| x \in I \} \text { tq } y_n \to \inf \{ |\partial f(x)|| x \in I \}
$$
Prenons \( (z_n) = (x_n/(2y_n)) \). Alors, on a bien
$$
(z_n) \subset
\left\{ \left. \left| \frac{\partial^2f(\xi)}{2\partial f(\eta)} \right| \right| \xi, \eta \in I \text{ et } \partial f(\eta) \neq 0 \right\}
$$
Enfin, par les propriétés des limites
$$
\lim_{n \to +\infty} z_n
= \frac{\lim_{n \to +\infty} x_n}{2 \lim_{n \to +\infty} y_n}
= \frac{\sup \{ | \partial^2 f(x) || x \in I \}}{2 \inf \{ |\partial f(x)|| x \in I \}}
$$
On a donc bien montré la propriété \eqref{eq:1.conv}, ce qui termine la preuve de l'exercice 1.

\newpage
\section{Exercice 2:}
\subsection{Partie a:}

\textbf{Objectif} : Trouver les racines de f : \( \mathbb{R} \rightarrow \mathbb{R} : x \mapsto x^4 - 8x^2 - 4 \).
Autrement dit, nous devons trouver les solutions de \( P \equiv x^4 - 8x^2 - 4 = 0 \). \newline
Tout d'abord, posons \( y = x^2 \).
On obtient : 

\begin{align*}
	Q &\equiv y^2 - 8y - 4 = 0\\ \\
	\vartriangle &= (-8)^2 - 4 * 1 * 4 \\
	       &= 64 + 16 \\
	       &= 80 \\ \\
   x_{1,2} &= \frac{-(-8) \pm \sqrt{80}}{2} \\
   \Rightarrow y_1 = &\frac{8 + \sqrt{80}}{2} ~ et ~ y_2 = \frac{8 - \sqrt{80}}{2} \\
   \Leftrightarrow y_1 = &\frac{8 + \sqrt{16 * 5}}{2} ~ et ~ y_2 = \frac{8 - \sqrt{16 * 5}}{2} \\
   \Leftrightarrow y_1 = &\frac{8 + 4 * \sqrt{5}}{2} ~ et ~ y_2 = \frac{8 - 4 * \sqrt{5}}{2} \\
    \Leftrightarrow y_1 = &4 + 2 * \sqrt{5} ~ et ~ y_2 = 4 - 2\sqrt{5}
\end{align*}


Donc, les racines de Q sont \( y_1 = 4 + 2 * \sqrt{5} ~ et ~ y_2 = 4 - 2 * \sqrt{5}  \) mais nous voulons les racines de P. Donc, on obtient : (on sait que y = $x^2$)


\begin{align*}
	y_1 = 4 + 2 * \sqrt{5} ~ et ~ y_2 = 4 - 2 * \sqrt{5} \\
	(x_1)^2 = 4 + 2 * \sqrt{5} ~ et ~ (x_2)^2 = 4 - 2 * \sqrt{5} \\
	Cependant, ~ (x_2)^2 = 4 - 2 * \sqrt{5} ~ est ~ impossible ~ car ~ on ~ a ~ : \\
	2 = \sqrt{4} \leq \sqrt{5} \leq \sqrt{9} = 3 ~ car ~ \surd ~ est ~ croissante \\
	Donc ~ 4 - 2 * \sqrt{5} \leq 4 - 2 * \sqrt{4} = 4 - 2 * 2 = 0 \\
	i.e. ~ (x_2)^2 = 4 - 2 * \sqrt{5} \leq 0 ~~~~~~~~~~~~~~~~~~~~~~~~~ \\
	Finalement, ~ les ~ racines ~ de ~ P ~ sont ~ : \\
	(x_1)^2 = 4 + 2 * \sqrt{5} \\
	\Leftrightarrow x_1 = \pm \sqrt{4 + 2 * \sqrt{5}}	
\end{align*}


\newpage
\subsection{Partie B:}

\begin{itemize}
	\item[\(\lceil 1 \rfloor\)] Montrons que f est paire.
	\item[\(\lceil 2 \rfloor\)] Montrons que f est croissante sur \( [ \sqrt{4 + 2 * \sqrt{5}} ; +\infty [ \)
	\item[\(\lceil 3 \rfloor\)] Montrons que f est décroissante sur \( ] -\infty ; -\sqrt{4 + 2 * \sqrt{5}} ] \)
	\item[\(\lceil 4 \rfloor\)] Montrer que f est positive sur \( [ \sqrt{4 + 2 * \sqrt{5}} ; +\infty [ \) et sur \( ] -\infty ; \sqrt{4 + 2 * \sqrt{5}} ] \)
	\item[\(\lceil 5 \rfloor\)] Montrer que f est négative sur \( [ -\sqrt{4 + 2 * \sqrt{5}} ; \sqrt{4 + 2 * \sqrt{5}} ] \)
\end{itemize}

\textbf{Preuve :}

\begin{itemize}
	\item[\(\lceil 1 \rfloor\)] trivial car une somme de fonction paire est paire.
	\item[\(\lceil 2 \rfloor\)] Montrons que \( \forall x, y \in [ \sqrt{4 + 2 * \sqrt{5}} ; +\infty [, ~ x \leq y \Rightarrow f(x) \leq f(y) \). \newline
	Soient x, y \( \in [ \sqrt{4 + 2 * \sqrt{5}} ; +\infty [ \), supposons \( x \leq y \). Montrons que \( f(x) \leq f(y) \). \newline On a :
	\begin{align*}
		f(x) \leq f(y) &\Leftrightarrow x^4 - 8x^2 - 4 \leq y^4 - 8y^2 - 4 \\
		&\Leftrightarrow x^4 - 8x^2 \leq y^4 - 8y^2 \\
		&\Leftrightarrow 0 \leq (y^4 - x^4) - (8y^2 - 8x^2) \\
		&\Leftrightarrow 0 \leq (y^4 - x^4) - 8(y^2 - x^2) \\
	\end{align*}
	ok car la fonction \( x^2 \) et \( x^4 \) sont croissantes sur \( \mathbb{R}^{+} \) et par hypothèse, \( x \leq y \).	 
	\item[\(\lceil 3 \rfloor\)] trivial car, par le fait que la fontion f est paire et croissante sur \( [ \sqrt{4 + 2 * \sqrt{5}} ; +\infty [ \), on a que f est décroissante sur \( ] -\infty ; -\sqrt{4 + 2 * \sqrt{5}} ] \).
	\item[\(\lceil 4 \rfloor\)] par la croissance sur \( [ \sqrt{4 + 2 * \sqrt{5}} ; +\infty [ \) ainsi que sa décroissance sur \( ] -\infty ; -\sqrt{4 + 2 * \sqrt{5}} ] \), on sait que \( \forall x \in ] -\infty ; -\sqrt{4 + 2 * \sqrt{5}} ] \cup [ \sqrt{4 + 2 * \sqrt{5}} ; +\infty [, f \left( \sqrt{4 + 2 * \sqrt{5}}\right) \leq f \left( x \right) \). Mais \( f \left( \sqrt{4 + 2 * \sqrt{5}} \right) = 0 \) car c'est une racine de f (pareil pour \( f \left( - \sqrt{4 + 2 * \sqrt{5}} \right) = 0 \) ). \newline
	Donc \( 0 = f \left( \sqrt{4 + 2 * \sqrt{5}} \right) \leq f \left( x \right) \) càd \( 0 \leq f \left( x \right) \). Donc f est positive sur \( ] -\infty ; -\sqrt{4 + 2 * \sqrt{5}} ] \cup [ \sqrt{4 + 2 * \sqrt{5}} ; +\infty [ \)
	\item[\(\lceil 5 \rfloor\)] Vu que la fonction est pair et que l'intervalle est de la forme ]-a; a[, alors on peut se restreindre à regarder l'intervalle \( [ 0 ; \sqrt{4 + 2 * \sqrt{5}} ] \).
	\newline \textbf{TODO}
	
\end{itemize}

\newpage
\subsection{Partie C:}
Pour la racine positive de $f$ ($x^{+*}=\sqrt{4+2\sqrt{5}}$),prenons l'intevalle $[\frac{5}{2},\frac{7}{2}]\ni x^{+*}$,\\
on a :
\begin{align*}
&f(\frac{5}{2})f(\frac{7}{2})<0 \\
\Leftrightarrow&\frac{-239}{16}\frac{769}{16}<0\\
\end{align*}
et $f$ est convexe sur $[\frac{5}{2},\frac{7}{2}]$.\\
Regardons l'intersection de la tangente au point $(\frac{5}{2},f(\frac{5}{2}))$ avec l'axe des abscisses,\\
\begin{align*}
&f(\frac{5}{2})+\partial f(\frac{5}{2})(x_1-\frac{5}{2})=0\\
\Leftrightarrow& x_1=\frac{1139}{360}\\
\end{align*}
De même pour la tangente au point $(\frac{7}{2},f(\frac{7}{2}))$,\\
\begin{align*}
&f(\frac{7}{2})+\partial f(\frac{7}{2})(x_2-\frac{7}{2})=0\\
\Leftrightarrow& x_2=\frac{5651}{1848}\\
\end{align*}
On a bien que $x_1,x_2\in[\frac{5}{2},\frac{7}{2}]$;\\
alors la suite $(x_n)_{n\in\mathbb{N}}$ construite par la méthode de Newton est bien définie (pour tout $n , \partial f(x_n)\neq 0$ et $x_n\in [\frac{5}{2},\frac{7}{2}]$) et converge vers l'unique racine de $f$ dans $[\frac{5}{2},\frac{7}{2}]$.\\
\\
Même raisonnement pour la racine négative de $f$ ($x^{-*}=-\sqrt{4+2\sqrt{5}}$) en prenant l'intervalle $[-\frac{7}{2},-\frac{5}{2}]\ni x^{-*}$, car $f$ est une fonction paire (sommme de fonctions paires).
\newpage
\section{Exercice 3:}
Soit \( f : [a, b] \rightarrow \mathbb{R} \) une application unimodale càd \( \exists m \in [a, b] \) tel que \emph{f est strictement croissante} sur [a, m] et \emph{strictement décroissante} sur [m, b].
\subsection{Partie a:}
Montrons que f possède au plus 2 racines. \newline \newline
Supposons par l'absurde que f possède 3 racines càd \newline 
\( \exists ~ u \neq v \neq w \in [a, b], ~ f(u) = f(v) = f(w) = 0 \). \newline
Sans perte de généralité, supposons que \( u < v < w \). \newline
On a donc 3 cas :
\begin{enumerate}
	\item[\(\lceil 1 \rfloor\)] soit \( u \in [a, m[ ~ et ~ v, w \in ]m, b] \)
	\item[\(\lceil 2 \rfloor\)] soit \( u, v \in [a, m[ ~ et ~ w \in ]m, b] \)
	\item[\(\lceil 3 \rfloor\)] soit u, v ou w est notre m.
\end{enumerate}
Montrons donc que tous les cas sont impossibles.
\begin{enumerate}
	\item[\(\lceil 1 \rfloor\)] on sait que \emph{f est strictement décroissante} sur [m, b] (et donc strictement décroissante sur ]m, b] par inclusion) donc \( v < w \Rightarrow f(v) > f(w) \) or ce sont 2 racines de f. \textbf{Contradiction}.
	\item[\(\lceil 2 \rfloor\)] on sait que \emph{f est strictement croissante} sur [a, m] (et donc strictement croissante sur [a, m[ par inclusion) donc \( u < v \Rightarrow f(u) < f(v) \) or ce sont 2 racines de f. \textbf{Contradiction}.
	\item[\(\lceil 3 \rfloor\)] $~$ \newline
	\begin{enumerate}
		\item[\(\lceil 3.1 \rfloor\)] Si u est m, alors \( u, v ~ et ~ w \in [m, b] \) et par \(\lceil 1 \rfloor\), c'est impossible. \textbf{Contradiction}.
		\item[\(\lceil 3.2 \rfloor\)] Si w est m, alors \( u, v ~ et ~ w \in [a, m] \) et par \(\lceil 2 \rfloor\), c'est impossible. \textbf{Contradiction}.
		\item[\(\lceil 3.3 \rfloor\)] Si v est m, alors \( \forall x \in [a, m[, ~ y \in ]m, b], ~ f(x) < f(v) \) (par la stricte croissance de f) et \( f(v) > f(y)\) (par la stricte décroissance de f). Cependant, en particulier, \( x = u \) et \( y = v \) et qui sont racines de f. Donc, \( 0 = f(u) < f(v = 0 \) et \( 0 = f(u) > f(w) = 0 \) qui sont tout 2 impossibles. \textbf{Contradiction}. 
	\end{enumerate}
\end{enumerate}

\newpage

On suppose maintenant pour le reste de la question 3 que \( f \in \mathcal{C}^{2}([a, b], \mathbb{R}) \). De plus, on suppose que toutes les racines de la dérivée de \( f \) sont simples.

\subsection{Partie b:}

On doit montrer que la dérivée de \( f \) possède une unique racine dans \( ]a, b[ \).

\subsubsection{Existance de la racine}

Montrons d'abord que la dérivée de \( f \) a une racine dans \( ]a, b[ \). Par l'hypothèse \( f \in \mathcal{C}^{2}([a, b], \mathbb{R}) \), on sait que \( \partial f \) est continue.
On sait également par unimodalité que
$$
\exists m \in ]a, b[,~ f \text{ est croissante sur } ]a, m[ \text{ et } f \text{ est décroissante sur } ]m, b[
$$
Comme la dérivée d'une fonction croissante (resp. décroissante) est positive (resp. négative), et comme \( (a+m)/2 \in ]a, m[ \) et \( (m+b)/2 \in ]m, b[ \), on sait que
$$
\partial f ((a+m)/2) \geq 0 \text{ et } \partial f((m+b)/2) \leq 0
$$
On peut alors séparer en 3 cas exhaustifs :
\begin{itemize}
    \item Cas 1 : \( f ((a+m)/2) = 0 \) \\
    Alors, comme \( (a+m)/2 \in ]a, m[ \subset ]a, b[ \), on a bien trouvé une racine de \( \partial f \) dans \( ]a, b[ \).
    \item Cas 2 : \( f ((m+b)/2) = 0 \) \\
    Alors, comme \( (m+b)/2 \in ]m, b[ \subset ]a, b[ \), on a bien trouvé une racine de \( \partial f \) dans \( ]a, b[ \).
    \item Cas 3 : \( \partial f ((a+m)/2) > 0 \) et \( \partial f((m+b)/2) < 0 \) \\
    Comme on sait que \( \partial f \) est continue, et comme \( \partial f ((a+m)/2) \partial f((m+b)/2) < 0 \), on peut appliquer le théorème des valeurs intermédiaires et on obtient
    $$
    \exists \xi \in \left] \frac{a+m}{2}, \frac{m+b}{2} \right[,~ \partial f (\xi) = 0
    $$
    Comme \( \xi \in ](a+m)/2, (m+b)/2[ \subset ]a, b[ \), on a bien trouvé une racine de \( \partial f \) dans \( ]a, b[ \).
\end{itemize}
Par exhaustivité des cas, on a bien montré l'existance d'une racine de \( \partial f \) dans \( ]a, b[ \).

\subsubsection{Unicité de la racine}

Il reste à montrer que cette racine est unique. Supposons par l'absurde que \( \partial f \) a au moins 2 racines distinctes càd
$$
\exists \xi_1 \neq \xi_2 \in ]a, b[,~ \partial f(\xi_1) = \partial f(\xi_2) = 0
$$
Supposons sans perte de généralité que \( \xi_1 < \xi_2 \).


\newpage
\subsection{Partie c:}

\begin{algorithm}

\caption{Recherche de racine}
\label{Modele pour un algo}

\begin{algorithmic}

\REQUIRE f, une fonction unimodale ; \( \partial f \), dérivée de f ; \( a, b  \in \mathbb{R} \).
\ENSURE L'ensemble des racines notée \( \varnothing \), \{ \( x_1 \) \} ou \{ \( x_1, x_2 \) \}.
\STATE \( fa \leftarrow f(a) \)
\STATE \( fb \leftarrow f(b) \)
\STATE \( dfa \leftarrow df(a) \)
\STATE \( dfb \leftarrow df(b) \)
\IF {\( fa \times fb > 0 \)}
\RETURN \( \varnothing \)
\ELSIF {\( fa \times fb = 0 \)}
\IF {\( fa = 0 ~ et ~ fb = 0 \)}
\RETURN \{ a, b \}
\ELSIF {\( fa = 0 \)}
\IF {\( fb > 0 \)}
\RETURN \{ a \}
\ELSE
\IF {\( dfa \leq 0 \)}
\RETURN \{ a \}
\ELSE
\STATE \( x \leftarrow \frac{|a-b|}{2} \)
\WHILE {\( f(x) < 0 \)}
\STATE \(x \leftarrow \frac{|a-x|}{2} \)
\ENDWHILE
\STATE \( root \leftarrow bissection(f, x, b) \)
\RETURN {a, root}
\ENDIF
\ENDIF
\ELSE
\IF {\( fa > 0 \)}
\RETURN \{ b \}
\ELSE
\IF {\( dfb \geq 0 \)}
\RETURN \{ a \}
\ELSE
\STATE \( x \leftarrow \frac{|a-b|}{2} \)
\WHILE {\( f(x) > 0 \)}
\STATE \(x \leftarrow \frac{|x-b|}{2} \)
\ENDWHILE
\STATE \( root \leftarrow bissection(f, a, x) \)
\RETURN {root, b}
\ENDIF
\ENDIF
\ENDIF
\ELSE
\STATE \( root \leftarrow bissection(f, a, b) \)
\RETURN {root}
\ENDIF

\end{algorithmic}

\end{algorithm}


\end{document}
