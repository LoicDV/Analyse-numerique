\documentclass[a4paper, 12pt]{article}
\usepackage[utf8]{inputenc}
\usepackage[pdftex]{graphicx}
\usepackage{tabularx}
\usepackage[french, english]{babel}
\usepackage{geometry}
\usepackage{pdfpages}
\usepackage{amsmath}
\usepackage{amssymb}
\usepackage{hyperref}
\geometry{hmargin=2cm,vmargin=1.5cm}
%\renewcommand{\baselinestretch}{0.96}

\begin{document}
\begin{titlepage}
\begin{center}

{\Large Université de Mons}\\[1ex]
{\Large Faculté des Sciences}\\[1ex]
{\Large Département des mathématiques}\\[2.5cm]

\newcommand{\HRule}{\rule{\linewidth}{0.3mm}}
% Title
\HRule \\[0.3cm]
{ \LARGE \bfseries Analyse Numérique : Rapport \\[0.3cm]}
{ \LARGE \bfseries TP 1 \\[0.1cm]}
\HRule \\[1.5cm]

% Author and supervisor
\begin{minipage}[t]{0.45\textwidth}
\begin{flushleft} \large
\emph{Professeurs:}\\
Christophe \textsc{Troestler} \\
Quentin \textsc{Lambotte}
\end{flushleft}
\end{minipage}
\begin{minipage}[t]{0.45\textwidth}
\begin{flushright} \large
\emph{Auteurs:} \\
Loïc \textsc{Dupont} \\
Maximilien \textsc{Vanhaverbeke} \\
Paulo \textsc{Marcelis}
\end{flushright}
\end{minipage}\\[2ex]

\vfill

% Bottom of the page
\begin{center}
\begin{tabular}[t]{c c c}
\includegraphics[height=1.5cm]{logoumons.jpg} &
\hspace{0.3cm} &
\includegraphics[height=1.5cm]{logofs.jpg}
\end{tabular}
\end{center}~\\
 
{\large Année académique 2020-2021}

\end{center}
\end{titlepage}

\newpage
\tableofcontents

\newpage
\section{Exercice 1:}

Soit \( I \) un intervalle. Soit \( f \in \mathcal{C}^2(I, \mathbb{R}) \). Supposons que \( \inf\{|\partial f(x)| | x \in I\} > 0 \). On doit montrer que
$$
\frac{\sup \{ | \partial^2 f(x) || x \in I \}}{2 \inf \{ |\partial f(x)|| x \in I \}}
= \sup \left\{ \left. \left| \frac{\partial^2f(\xi)}{2\partial f(\eta)} \right| \right| \xi, \eta \in I \text{ et } \partial f(\eta) \neq 0 \right\}
$$
càd
\begin{equation}
    \label{eq:1.maj}
    \frac{\sup \{ | \partial^2 f(x) || x \in I \}}{2 \inf \{ |\partial f(x)|| x \in I \}} \text{ est un majorant de }
    \left\{ \left. \left| \frac{\partial^2f(\xi)}{2\partial f(\eta)} \right| \right| \xi, \eta \in I \text{ et } \partial f(\eta) \neq 0 \right\}
\end{equation}
et
\begin{equation}
    \label{eq:1.conv}
    \exists (z_n) \subset
    \left\{ \left. \left| \frac{\partial^2f(\xi)}{2\partial f(\eta)} \right| \right| \xi, \eta \in I \text{ et } \partial f(\eta) \neq 0 \right\}
    \text{ tq } z_n \to
    \frac{\sup \{ | \partial^2 f(x) || x \in I \}}{2 \inf \{ |\partial f(x)|| x \in I \}}
\end{equation}

\subsection{Preuve de la majoration}

% \eqref{eq:1.maj} Juste pour ne pas perdre la commande, à supprimer
Comme un supremum est en particulier un majorant, on sait que
\begin{equation}
    \label{eq:1.sup}
    \forall \xi \in I,~ \sup \{ | \partial^2 f(x) || x \in I \} \geq |\partial^2 f(\xi)|
\end{equation}
De même, un infimum étant en particulier un minorant, on sait que
$$
\forall \eta \in I \text{ tq } \partial f(\eta) \neq 0,~ \inf \{ |\partial f(x)|| x \in I \} \leq |\partial f(\eta)|
$$
Comme \( \inf \{ |\partial f(x)|| x \in I \} > 0 \) par hypothèse, \( \inf \{ |\partial f(x)|| x \in I \} \in \mathbb{R}^{>0} \). On sait aussi que 
\(\forall \eta \in I \text{ tq } \partial f(\eta) \neq 0,~ |\partial f(\eta)| \in \mathbb{R}^{>0} \).
Donc, par décroissance de la fonction \( x \mapsto 1/x \) sur \( \mathbb{R}^{>0} \), on a
\begin{equation}
    \label{eq:1.inf}
    \forall \eta \in I \text{ tq } \partial f(\eta) \neq 0,~ \frac{1}{\inf \{ |\partial f(x)|| x \in I \}} \geq \frac{1}{|\partial f(\eta)|}
\end{equation}
Finalement, par \eqref{eq:1.sup} et \eqref{eq:1.inf}, on a
$$
\forall \xi, \eta \in I \text{ tq } \partial f(\eta) \neq 0,~ \frac{\sup \{ | \partial^2 f(x) || x \in I \}}{2 \inf \{ |\partial f(x)|| x \in I \}}
\geq \frac{| \partial^2f(\xi)|}{2|\partial f(\eta)|}
= \left| \frac{\partial^2f(\xi)}{2\partial f(\eta)} \right|
$$
On a donc bien montré la propriété \eqref{eq:1.maj}.

\subsection{Preuve de la suite convergente}

Par les propriétés du supremum, on sait que
$$
\exists (x_n) \subset \{ | \partial^2 f(x) || x \in I \} \text{ tq } x_n \to \sup \{ | \partial^2 f(x) || x \in I \}
$$
De même, par les propriétés de l'infimum, on sait que
$$
\exists (y_n) \subset \{ |\partial f(x)|| x \in I \} \text { tq } y_n \to \inf \{ |\partial f(x)|| x \in I \}
$$
Prenons \( (z_n) = (x_n/(2y_n)) \). Alors, on a bien
$$
(z_n) \subset
\left\{ \left. \left| \frac{\partial^2f(\xi)}{2\partial f(\eta)} \right| \right| \xi, \eta \in I \text{ et } \partial f(\eta) \neq 0 \right\}
$$
Enfin, par les propriétés des limites
$$
\lim_{n \to +\infty} z_n
= \frac{\lim_{n \to +\infty} x_n}{2 \lim_{n \to +\infty} y_n}
= \frac{\sup \{ | \partial^2 f(x) || x \in I \}}{2 \inf \{ |\partial f(x)|| x \in I \}}
$$
On a donc bien montré la propriété \eqref{eq:1.conv}, ce qui termine la preuve de l'exercice 1.

\newpage
\section{Exercice 2:}
\subsection{Partie a:}

\textbf{Objectif} : Trouver les racines de f : \( \mathbb{R} \rightarrow \mathbb{R} : x \mapsto x^4 - 8x^2 - 4 \).
Autrement dit, nous devons trouver les solutions de \( P \equiv x^4 - 8x^2 - 4 = 0 \). \newline
Tout d'abord, posons \( y = x^2 \).
On obtient : 

\begin{align*}
	Q &\equiv y^2 - 8y - 4 = 0\\ \\
	\vartriangle &= (-8)^2 - 4 * 1 * 4 \\
	       &= 64 + 16 \\
	       &= 80 \\ \\
   x_{1,2} &= \frac{-(-8) \pm \sqrt{80}}{2} \\
   \Rightarrow y_1 = &\frac{8 + \sqrt{80}}{2} ~ et ~ y_2 = \frac{8 - \sqrt{80}}{2} \\
   \Leftrightarrow y_1 = &\frac{8 + \sqrt{16 * 5}}{2} ~ et ~ y_2 = \frac{8 - \sqrt{16 * 5}}{2} \\
   \Leftrightarrow y_1 = &\frac{8 + 4 * \sqrt{5}}{2} ~ et ~ y_2 = \frac{8 - 4 * \sqrt{5}}{2} \\
    \Leftrightarrow y_1 = &4 + 2 * \sqrt{5} ~ et ~ y_2 = 4 - 2\sqrt{5}
\end{align*}


Donc, les racines de Q sont \( y_1 = 4 + 2 * \sqrt{5} ~ et ~ y_2 = 4 - 2 * \sqrt{5}  \) mais nous voulons les racines de P. Donc, on obtient : (on sait que y = $x^2$)


\begin{align*}
	y_1 = 4 + 2 * \sqrt{5} ~ et ~ y_2 = 4 - 2 * \sqrt{5} \\
	(x_1)^2 = 4 + 2 * \sqrt{5} ~ et ~ (x_2)^2 = 4 - 2 * \sqrt{5} \\
	Cependant, ~ (x_2)^2 = 4 - 2 * \sqrt{5} ~ est ~ impossible ~ car ~ on ~ a ~ : \\
	2 = \sqrt{4} \leq \sqrt{5} \leq \sqrt{9} = 3 ~ car ~ \surd ~ est ~ croissante \\
	Donc ~ 4 - 2 * \sqrt{5} \leq 4 - 2 * \sqrt{4} = 4 - 2 * 2 = 0 \\
	i.e. ~ (x_2)^2 = 4 - 2 * \sqrt{5} \leq 0 ~~~~~~~~~~~~~~~~~~~~~~~~~ \\
	Finalement, ~ les ~ racines ~ de ~ P ~ sont ~ : \\
	(x_1)^2 = 4 + 2 * \sqrt{5} \\
	\Leftrightarrow x_1 = \pm \sqrt{4 + 2 * \sqrt{5}}	
\end{align*}


\newpage
\subsection{Partie B:}

\begin{itemize}
	\item[\(\lceil 1 \rfloor\)] Montrons que f est paire.
	\item[\(\lceil 2 \rfloor\)] Montrons que f est croissante sur \( [ \sqrt{4 + 2 * \sqrt{5}} ; +\infty [ \)
	\item[\(\lceil 3 \rfloor\)] Montrons que f est décroissante sur \( ] -\infty ; -\sqrt{4 + 2 * \sqrt{5}} ] \)
	\item[\(\lceil 4 \rfloor\)] Montrer que f est positive sur \( [ \sqrt{4 + 2 * \sqrt{5}} ; +\infty [ \) et sur \( ] -\infty ; \sqrt{4 + 2 * \sqrt{5}} ] \)
	\item[\(\lceil 5 \rfloor\)] Montrer que f est négative sur \( [ -\sqrt{4 + 2 * \sqrt{5}} ; \sqrt{4 + 2 * \sqrt{5}} ] \)
\end{itemize}

\textbf{Preuve :}

\begin{itemize}
	\item[\(\lceil 1 \rfloor\)] trivial car une somme de fonction paire est paire.
	\item[\(\lceil 2 \rfloor\)] Montrons que \( \forall x, y \in [ \sqrt{4 + 2 * \sqrt{5}} ; +\infty [, ~ x \leq y \Rightarrow f(x) \leq f(y) \). \newline
	Soient x, y \( \in [ \sqrt{4 + 2 * \sqrt{5}} ; +\infty [ \), supposons \( x \leq y \). Montrons que \( f(x) \leq f(y) \). \newline On a :
	\begin{align*}
		f(x) \leq f(y) &\Leftrightarrow x^4 - 8x^2 - 4 \leq y^4 - 8y^2 - 4 \\
		&\Leftrightarrow x^4 - 8x^2 \leq y^4 - 8y^2 \\
		&\Leftrightarrow 0 \leq (y^4 - x^4) - (8y^2 - 8x^2) \\
		&\Leftrightarrow 0 \leq (y^4 - x^4) - 8(y^2 - x^2) \\
	\end{align*}
	ok car la fonction \( x^2 \) et \( x^4 \) sont croissantes sur \( \mathbb{R}^{+} \) et par hypothèse, \( x \leq y \).	 
	\item[\(\lceil 3 \rfloor\)] trivial car, par le fait que la fontion f est paire et croissante sur \( [ \sqrt{4 + 2 * \sqrt{5}} ; +\infty [ \), on a que f est décroissante sur \( ] -\infty ; -\sqrt{4 + 2 * \sqrt{5}} ] \).
	\item[\(\lceil 4 \rfloor\)] par la croissance sur \( [ \sqrt{4 + 2 * \sqrt{5}} ; +\infty [ \) ainsi que sa décroissance sur \( ] -\infty ; -\sqrt{4 + 2 * \sqrt{5}} ] \), on sait que \( \forall x \in ] -\infty ; -\sqrt{4 + 2 * \sqrt{5}} ] \cup [ \sqrt{4 + 2 * \sqrt{5}} ; +\infty [, f \left( \sqrt{4 + 2 * \sqrt{5}}\right) \leq f \left( x \right) \). Mais \( f \left( \sqrt{4 + 2 * \sqrt{5}} \right) = 0 \) car c'est une racine de f (pareil pour \( f \left( - \sqrt{4 + 2 * \sqrt{5}} \right) = 0 \) ). \newline
	Donc \( 0 = f \left( \sqrt{4 + 2 * \sqrt{5}} \right) \leq f \left( x \right) \) càd \( 0 \leq f \left( x \right) \). Donc f est positive sur \( ] -\infty ; -\sqrt{4 + 2 * \sqrt{5}} ] \cup [ \sqrt{4 + 2 * \sqrt{5}} ; +\infty [ \)
	\item[\(\lceil 5 \rfloor\)] Vu que la fonction est pair et que l'intervalle est de la forme ]-a; a[, alors on peut se restreindre à regarder l'intervalle \( [ 0 ; \sqrt{4 + 2 * \sqrt{5}} ] \).
	\newline \textbf{TODO}
	
\end{itemize}

\newpage
\subsection{Partie C:}
Je te laisse compléter Paolo ;)


\end{document}
