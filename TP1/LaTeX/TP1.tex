\documentclass[a4paper, 12pt]{article}
\usepackage[utf8]{inputenc}
\usepackage[pdftex]{graphicx}
\usepackage{tabularx}
\usepackage[french, english]{babel}
\usepackage{geometry}
\usepackage{pdfpages}
\usepackage{amsmath}
\usepackage{amssymb}
\usepackage{hyperref}
\usepackage{algorithmic}
\usepackage[french,linesnumbered,algoruled]{algorithm2e}

\newcommand{\EFFECT}{\textbf{Effet : }}
\renewcommand{\algorithmicrequire}{\textbf{Entrée :}}
\renewcommand{\algorithmicensure}{\textbf{Sortie :}}
\renewcommand{\algorithmicend}{\textbf{fin}}
\renewcommand{\algorithmicif}{\textbf{si}}
\renewcommand{\algorithmicthen}{\textbf{alors}}
\renewcommand{\algorithmicelse}{\textbf{sinon}}
\renewcommand{\algorithmicelsif}{\algorithmicelse\ \algorithmicif}
\renewcommand{\algorithmicendif}{\algorithmicend\ \algorithmicif}
\renewcommand{\algorithmicfor}{\textbf{pour}}
\renewcommand{\algorithmicforall}{\textbf{pour tout}}
\renewcommand{\algorithmicdo}{\textbf{faire}}
\renewcommand{\algorithmicendfor}{\algorithmicend\ \algorithmicfor}
\renewcommand{\algorithmicwhile}{\textbf{tant que}}
\renewcommand{\algorithmicendwhile}{\algorithmicend\ \algorithmicwhile}
\renewcommand{\algorithmicloop}{\textbf{boucle}}
\renewcommand{\algorithmicendloop}{\algorithmicend\ \algorithmicloop}
\renewcommand{\algorithmicrepeat}{\textbf{répéter}}
\renewcommand{\algorithmicuntil}{\textbf{jusqu'à}}
\renewcommand{\algorithmicprint}{\textbf{imprimer}}
\renewcommand{\algorithmicreturn}{\textbf{retourner}}
\renewcommand{\algorithmictrue}{\textbf{vrai}}
\renewcommand{\algorithmicfalse}{\textbf{faux}}
\makeatletter
\renewcommand{\algocf@captiontext}[2]{\AlCapNameSty{\centerline{\AlCapNameFnt{}#2}}}
\makeatother

\geometry{hmargin=2cm,vmargin=1.5cm}
%\renewcommand{\baselinestretch}{0.96}

\begin{document}
\begin{titlepage}
\begin{center}

{\Large Université de Mons}\\[1ex]
{\Large Faculté des Sciences}\\[1ex]
{\Large Département des mathématiques}\\[2.5cm]

\newcommand{\HRule}{\rule{\linewidth}{0.3mm}}
% Title
\HRule \\[0.3cm]
{ \LARGE \bfseries Analyse Numérique : Rapport \\[0.3cm]}
{ \LARGE \bfseries TP 1 \\[0.1cm]}
\HRule \\[1.5cm]

% Author and supervisor
\begin{minipage}[t]{0.45\textwidth}
\begin{flushleft} \large
\emph{Professeurs:}\\
Christophe \textsc{Troestler} \\
Quentin \textsc{Lambotte}
\end{flushleft}
\end{minipage}
\begin{minipage}[t]{0.45\textwidth}
\begin{flushright} \large
\emph{Auteurs:} \\
Loïc \textsc{Dupont} \\
Maximilien \textsc{Vanhaverbeke} \\
Paolo \textsc{Marcelis}
\end{flushright}
\end{minipage}\\[2ex]

\vfill

% Bottom of the page
\begin{center}
\begin{tabular}[t]{c c c}
\includegraphics[height=1.5cm]{logoumons.jpg} &
\hspace{0.3cm} &
\includegraphics[height=1.5cm]{logofs.jpg}
\end{tabular}
\end{center}~\\

{\large Année académique 2020-2021}

\end{center}
\end{titlepage}

\newpage
\renewcommand{\contentsname}{Table des matières}
\tableofcontents

\newpage

\section{Exercice 1:}

Soit \( I \) un intervalle compact non-vide. Soit \( f \in \mathcal{C}^2(I, \mathbb{R}) \). Supposons que \( \inf\{|\partial f(x)| | x \in I\} > 0 \). On doit montrer que
$$
\frac{\sup \{ | \partial^2 f(x) || x \in I \}}{2 \inf \{ |\partial f(x)|| x \in I \}}
= \sup \left\{ \left. \left| \frac{\partial^2f(\xi)}{2\partial f(\eta)} \right| \right| \xi, \eta \in I \text{ et } \partial f(\eta) \neq 0 \right\}
$$
càd
\begin{equation}
    \label{eq:1.maj}
    \frac{\sup \{ | \partial^2 f(x) || x \in I \}}{2 \inf \{ |\partial f(x)|| x \in I \}} \text{ est un majorant de }
    \left\{ \left. \left| \frac{\partial^2f(\xi)}{2\partial f(\eta)} \right| \right| \xi, \eta \in I \text{ et } \partial f(\eta) \neq 0 \right\}
\end{equation}
et
\begin{equation}
    \label{eq:1.conv}
    \exists (z_n) \subset
    \left\{ \left. \left| \frac{\partial^2f(\xi)}{2\partial f(\eta)} \right| \right| \xi, \eta \in I \text{ et } \partial f(\eta) \neq 0 \right\}
    \text{ tq } z_n \to
    \frac{\sup \{ | \partial^2 f(x) || x \in I \}}{2 \inf \{ |\partial f(x)|| x \in I \}}
\end{equation}

\subsection{Remarque sur l'existance du membre de gauche}

On sait par l'hypothèse \( f \in \mathcal{C}^2(I, \mathbb{R}) \) que \( \partial f \) et \( \partial^2 f \) sont des fonctions continues sur \( I \). Comme la valeur absolue est aussi une fonction continue, par composition de fonctions continues, \( | \partial f | \) et \( | \partial^2 f | \) sont des aussi fonctions continues sur \( I \). \\
Alors, comme \( I \) est un compact non-vide par hypothèse, le théorème des bornes atteintes nous dit que ces fonctions atteignent leurs bornes. En particulier,
$$
\sup \{ | \partial^2 f(x) || x \in I \} \in \mathbb{R} \quad \text{et} \quad
\inf \{ |\partial f(x)|| x \in I \} \in \mathbb{R}
$$

\subsection{Preuve de la majoration}

Comme un supremum est en particulier un majorant, on sait que
\begin{equation}
    \label{eq:1.sup}
    \forall \xi \in I,~ \sup \{ | \partial^2 f(x) || x \in I \} \geq |\partial^2 f(\xi)|
\end{equation}
De même, un infimum étant en particulier un minorant, on sait que
$$
\forall \eta \in I \text{ tq } \partial f(\eta) \neq 0,~ \inf \{ |\partial f(x)|| x \in I \} \leq |\partial f(\eta)|
$$
Comme \( \inf \{ |\partial f(x)|| x \in I \} > 0 \) par hypothèse, \( \inf \{ |\partial f(x)|| x \in I \} \in \mathbb{R}^{>0} \). On sait aussi que
\(\forall \eta \in I \text{ tq } \partial f(\eta) \neq 0,~ |\partial f(\eta)| \in \mathbb{R}^{>0} \).
Donc, par décroissance de la fonction \( x \mapsto 1/x \) sur \( \mathbb{R}^{>0} \), on a
\begin{equation}
    \label{eq:1.inf}
    \forall \eta \in I \text{ tq } \partial f(\eta) \neq 0,~ \frac{1}{\inf \{ |\partial f(x)|| x \in I \}} \geq \frac{1}{|\partial f(\eta)|}
\end{equation}
Finalement, par \eqref{eq:1.sup} et \eqref{eq:1.inf}, on a
$$
\forall \xi, \eta \in I \text{ tq } \partial f(\eta) \neq 0,~ \frac{\sup \{ | \partial^2 f(x) || x \in I \}}{2 \inf \{ |\partial f(x)|| x \in I \}}
\geq \frac{| \partial^2f(\xi)|}{2|\partial f(\eta)|}
= \left| \frac{\partial^2f(\xi)}{2\partial f(\eta)} \right|
$$
On a donc bien montré la propriété \eqref{eq:1.maj}.

\newpage

\subsection{Preuve de la suite convergente}

Par les propriétés du supremum, on sait que
$$
\exists (x_n) \subset \{ | \partial^2 f(x) || x \in I \} \text{ tq } x_n \to \sup \{ | \partial^2 f(x) || x \in I \}
$$
De même, par les propriétés de l'infimum, on sait que
$$
\exists (y_n) \subset \{ |\partial f(x)|| x \in I \} \text { tq } y_n \to \inf \{ |\partial f(x)|| x \in I \}
$$
Remarquons que par hypothèse, \( \inf\{|\partial f(x)| | x \in I\} > 0 \). Comme un infimum est en particulier un minorant, on sait que
$$
\forall \eta \in I,~ |\partial f(\eta)| \geq \inf\{|\partial f(x)| | x \in I\} > 0 \quad \text{et donc} \quad \forall \eta \in I,~ \partial f(\eta) \neq 0
$$
En particulier, la suite \( (y_n) \) ne s'annule donc jamais. Prenons \( (z_n) = (x_n/(2y_n)) \). Alors, on a bien
$$
(z_n) \subset
\left\{ \left. \left| \frac{\partial^2f(\xi)}{2\partial f(\eta)} \right| \right| \xi, \eta \in I \text{ et } \partial f(\eta) \neq 0 \right\}
$$
Enfin, par les propriétés des limites
$$
\lim_{n \to +\infty} z_n
= \frac{\lim_{n \to +\infty} x_n}{2 \lim_{n \to +\infty} y_n}
= \frac{\sup \{ | \partial^2 f(x) || x \in I \}}{2 \inf \{ |\partial f(x)|| x \in I \}}
$$
On a donc bien montré la propriété \eqref{eq:1.conv}, ce qui termine la preuve de l'exercice 1.

\newpage

\section{Exercice 2:}

\subsection{Partie A:}

\textbf{Objectif} : Trouver les racines de f : \( \mathbb{R} \rightarrow \mathbb{R} : x \mapsto x^4 - 8x^2 - 4 \).
Autrement dit, nous devons trouver les solutions de \( P \equiv x^4 - 8x^2 - 4 = 0 \). \newline
Tout d'abord, posons \( y = x^2 \).
On obtient :

\begin{align*}
	Q &\equiv y^2 - 8y - 4 = 0\\ \\
	\vartriangle &= (-8)^2 - 4 * 1 * 4 \\
	       &= 64 + 16 \\
	       &= 80 \\ \\
   y_{1,2} &= \frac{-(-8) \pm \sqrt{80}}{2} \\
   \Rightarrow y_1 = &\frac{8 + \sqrt{80}}{2} ~ et ~ y_2 = \frac{8 - \sqrt{80}}{2} \\
   \Leftrightarrow y_1 = &\frac{8 + \sqrt{16 * 5}}{2} ~ et ~ y_2 = \frac{8 - \sqrt{16 * 5}}{2} \\
   \Leftrightarrow y_1 = &\frac{8 + 4 * \sqrt{5}}{2} ~ et ~ y_2 = \frac{8 - 4 * \sqrt{5}}{2} \\
    \Leftrightarrow y_1 = &4 + 2 * \sqrt{5} ~ et ~ y_2 = 4 - 2\sqrt{5}
\end{align*}


Donc, les racines de Q sont \( y_1 = 4 + 2 * \sqrt{5} ~ et ~ y_2 = 4 - 2 * \sqrt{5}  \) mais nous voulons les racines de P. Donc, on obtient (on sait que y = $x^2$):


\begin{align*}
	y_1 = 4 + 2 * \sqrt{5}&\text{ et }y_2 = 4 - 2 * \sqrt{5} \\
	(x_{1,2})^2 = 4 + 2 * \sqrt{5}&\text{ et }(x_{3,4})^2 = 4 - 2 * \sqrt{5}
\end{align*}

	Cependant, $(x_{3,4})^2 = 4 - 2 * \sqrt{5}$ n'a pas de solution dans \( \mathbb{R} \) car on a :

\begin{align*}
	2 = \sqrt{4} \leq \sqrt{5} \leq \sqrt{9} = 3 \text{ car la racine carrée est croissante} \\
	\text{Donc } 4 - 2 * \sqrt{5} \leq 4 - 2 * \sqrt{4} = 4 - 2 * 2 = 0 \\
	i.e. ~ (x_{3,4})^2 = 4 - 2 * \sqrt{5} \leq 0
\end{align*}
Finalement, les racines de $P$ sont : \\
\begin{align*}
	(x_{1,2})^2 = 4 + 2 * \sqrt{5} \\
	\Leftrightarrow x_{1,2} = \pm \sqrt{4 + 2 * \sqrt{5}}
\end{align*}


\newpage

\subsection{Partie B:}

\begin{itemize}
	\item[\(\lceil 1 \rfloor\)] Montrons que f est paire.
	\item[\(\lceil 2 \rfloor\)] Montrons que f est croissante sur \( [ \sqrt{4 + 2 * \sqrt{5}} ; +\infty [ \)
	\item[\(\lceil 3 \rfloor\)] Montrons que f est décroissante sur \( ] -\infty ; -\sqrt{4 + 2 * \sqrt{5}} ] \)
	\item[\(\lceil 4 \rfloor\)] Montrons que f est positive sur \( [ \sqrt{4 + 2 * \sqrt{5}} ; +\infty [ \) et sur \( ] -\infty ; -\sqrt{4 + 2 * \sqrt{5}} ] \)
	\item[\(\lceil 5 \rfloor\)] Montrons que f est négative sur \( [ -\sqrt{4 + 2 * \sqrt{5}} ; \sqrt{4 + 2 * \sqrt{5}} ] \)
\end{itemize}

\textbf{Preuve :}

\begin{itemize}
	\item[\(\lceil 1 \rfloor\)] trivial car une somme de fonctions paires est paire.
	\item[\(\lceil 2 \rfloor\)] Montrons que \( \forall x, y \in [ \sqrt{4 + 2 * \sqrt{5}} ; +\infty [, ~ x \leq y \Rightarrow f(x) \leq f(y) \). \newline
	Soient x, y \( \in [ \sqrt{4 + 2 * \sqrt{5}} ; +\infty [ \), supposons \( x \leq y \). Montrons que \( f(x) \leq f(y) \). \newline On a :
	\begin{align*}
		f(x) \leq f(y) &\Leftrightarrow x^4 - 8x^2 - 4 \leq y^4 - 8y^2 - 4 \\
		&\Leftrightarrow x^4 - 8x^2 \leq y^4 - 8y^2 \\
		&\Leftrightarrow 0 \leq (y^4 - x^4) - (8y^2 - 8x^2) \\
		&\Leftrightarrow 0 \leq (y^4 - x^4) - 8(y^2 - x^2) \\
		&\Leftrightarrow 8 \leq (y^2 + x^2) &&\text{pour y $\neq$ x} \\
	\end{align*}
Vrai car vrai pour $ y = x = \sqrt{4+2*\sqrt{5}}$ et vrai pour toutes valeurs plus grandes car la fonction $x^2$ est croissante sur $\mathbb{R}^+$. Pour $x=y$ , trivial car $f(x)=f(y)$.
	\item[\(\lceil 3 \rfloor\)] trivial car, par le fait que la fontion f est paire et croissante sur \( [ \sqrt{4 + 2 * \sqrt{5}} ; +\infty [ \), on a que f est décroissante sur \( ] -\infty ; -\sqrt{4 + 2 * \sqrt{5}} ] \).
	\item[\(\lceil 4 \rfloor\)] par la croissance sur \( [ \sqrt{4 + 2 * \sqrt{5}} ; +\infty [ \) ainsi que sa décroissance sur \( ] -\infty ; -\sqrt{4 + 2 * \sqrt{5}} ] \), on sait que \( \forall x \in ] -\infty ; -\sqrt{4 + 2 * \sqrt{5}} ] \cup [ \sqrt{4 + 2 * \sqrt{5}} ; +\infty [, f \left( \sqrt{4 + 2 * \sqrt{5}}\right) \leq f \left( x \right) \) (car $f$ est paire). Mais \( f \left( \sqrt{4 + 2 * \sqrt{5}} \right) = 0 \) car c'est une racine de f (pareil pour \( f \left( - \sqrt{4 + 2 * \sqrt{5}} \right) = 0 \) ). \newline
	Donc \( 0 = f \left( \sqrt{4 + 2 * \sqrt{5}} \right) \leq f \left( x \right) \) càd \( 0 \leq f \left( x \right) \). Donc f est positive sur \( ] -\infty ; -\sqrt{4 + 2 * \sqrt{5}} ] \cup [ \sqrt{4 + 2 * \sqrt{5}} ; +\infty [ \)
	\item[\(\lceil 5 \rfloor\)] Montrons que $f$ est négative sur $ [ -\sqrt{4 + 2 * \sqrt{5}} ; \sqrt{4 + 2 * \sqrt{5}} ] $
\begin{itemize}
	\item[$\bullet$] Montrons que $-\sqrt{4 + 2 * \sqrt{5}}$ et $\sqrt{4 + 2 * \sqrt{5}}$ sont des racines simples :\\
	$\partial f(\sqrt{4 + 2 * \sqrt{5}}) > 0 $ et $\partial f(-\sqrt{4 + 2 * \sqrt{5}}) < 0$ donc ce sont des racines simples.
	\item[$\bullet$] On sait que $f$ est continue, ses racines sont simples et $f$ est positive sur \( ] -\infty ; -\sqrt{4 + 2 * \sqrt{5}} ] \cup [ \sqrt{4 + 2 * \sqrt{5}} ; +\infty [ \) donc $f$ est négative sur $ [ -\sqrt{4 + 2 * \sqrt{5}} ; \sqrt{4 + 2 * \sqrt{5}} ] $ car il n'y a pas de racines dans $ ] -\sqrt{4 + 2 * \sqrt{5}} ; \sqrt{4 + 2 * \sqrt{5}} [ $.
\end{itemize}

\end{itemize}
Finalement, pour la racine $-\sqrt{4 + 2 * \sqrt{5}}$, tout intervalle $[a,b]$, avec $a\in ]+\infty,-\sqrt{4 + 2 * \sqrt{5}}[$ et $ b\in ] -\sqrt{4 + 2 * \sqrt{5}} ; \sqrt{4 + 2 * \sqrt{5}} [$, est valide pour la méthode de bissection. De même pour la racine $\sqrt{4 + 2 * \sqrt{5}}$ , tout intervalle $[a,b]$, avec $a\in ] -\sqrt{4 + 2 * \sqrt{5}} ; \sqrt{4 + 2 * \sqrt{5}} [$ et $b\in ]\sqrt{4 + 2 * \sqrt{5}},+\infty[$, est valide pour la méthode de bissection.
\newpage

\subsection{Partie C:}
Pour la racine positive de $f$ ($x^{+*}=\sqrt{4+2\sqrt{5}}$),prenons l'intevalle $[\frac{5}{2},\frac{7}{2}]\ni x^{+*}$,\\
on a :
\begin{align*}
&f(\frac{5}{2})f(\frac{7}{2})<0 \\
\Leftrightarrow&\frac{-239}{16}\frac{769}{16}<0\\
\end{align*}
et $f$ est convexe sur $[\frac{5}{2},\frac{7}{2}]$.($\star$)(voir en fin d'exercice)\\
Regardons l'intersection de la tangente au point $(\frac{5}{2},f(\frac{5}{2}))$ avec l'axe des abscisses,\\
\begin{align*}
&f(\frac{5}{2})+\partial f(\frac{5}{2})(x_1-\frac{5}{2})=0\\
\Leftrightarrow& x_1=\frac{1139}{360}\\
\end{align*}
De même pour la tangente au point $(\frac{7}{2},f(\frac{7}{2}))$,\\
\begin{align*}
&f(\frac{7}{2})+\partial f(\frac{7}{2})(x_2-\frac{7}{2})=0\\
\Leftrightarrow& x_2=\frac{5651}{1848}\\
\end{align*}
On a bien que $x_1,x_2\in[\frac{5}{2},\frac{7}{2}]$;\\
alors la suite $(x_n)_{n\in\mathbb{N}}$ construite par la méthode de Newton est bien définie (pour tout $n , \partial f(x_n)\neq 0$ et $x_n\in [\frac{5}{2},\frac{7}{2}]$) et converge vers l'unique racine de $f$ dans $[\frac{5}{2},\frac{7}{2}]$.\\
\\
Même raisonnement pour la racine négative de $f$ ($x^{-*}=-\sqrt{4+2\sqrt{5}}$) en prenant l'intervalle $[-\frac{7}{2},-\frac{5}{2}]\ni x^{-*}$, car $f$ est une fonction paire (sommme de fonctions paires).\\[10mm]($\star$)

\begin{itemize}
	\item[$\bullet$] Montrons que $\partial f $ est croissante sur $[\frac{5}{2},\frac{7}{2}]$\\
		Soit $x,y \in [\frac{5}{2},\frac{7}{2}]$.\\
		Supposons $x \leqslant y$.
		Montrons que $\partial f(x) \leqslant \partial f(y)$.
		On a :
		\begin{align*}
		\partial f(x) \leqslant \partial f(y)& \Leftrightarrow \partial_x (x^4-8x^2-4) \leqslant \partial_y (y^4-8y^2-4)\\
		& \Leftrightarrow 4x^3-16x \leqslant 4y^3-16y\\
		& \Leftrightarrow 4x(x^2-4)\leqslant 4y(y^2-4)\\
		& \Leftrightarrow x(x^2-4) \leqslant y(y^2-4)\\
		\text{Comme }x,y\in \left[ \frac{5}{2},\frac{7}{2} \right] & \subset\mathbb{R}^+ \text{ et } x \leq y,~ \text{ on peut juste montrer que :} \\
		&(x^2-4) \leqslant (y^2-4)\\
		&\Leftrightarrow x^2\leqslant y^2 \text{ ok car par hypothèse } x\leqslant y \Rightarrow x^2 \leqslant y^2 \text{ car } x,y\in \left[\frac{5}{2},\frac{7}{2}\right] \\ \text{ et } x^2 \text{ est croissante sur  }\mathbb{R}^+.
		\end{align*}
	Donc $\partial f(x) \leqslant \partial f(y)$.
	\item[$\bullet$] Montrons que $f$ est convexe sur $[\frac{5}{2},\frac{7}{2}]$\\
	Vrai car si $\partial f$ est croissante sur $[a,b]$ cela implique que $f$ est convexe sur $[a,b]$ (propriété du syllabus)
\end{itemize}

\newpage

\section{Exercice 3:}

Soit \( f : [a, b] \rightarrow \mathbb{R} \) une application unimodale càd \( \exists m \in [a, b] \) tel que \emph{f est strictement croissante} sur [a, m] et \emph{strictement décroissante} sur [m, b].

\subsection{Partie A:}

Montrons que f possède au plus 2 racines. \newline \newline
Supposons par l'absurde que f possède au moins 3 racines càd \newline
\( \exists ~ u \neq v \neq w \in [a, b], ~ f(u) = f(v) = f(w) = 0 \). \newline
Sans perte de généralité, supposons que \( u < v < w \). \newline
On a donc 3 cas :
\begin{enumerate}
	\item[\(\lceil 1 \rfloor\)] soit \( u \in [a, m[ ~ et ~ v, w \in ]m, b] \) (mêmes arguments pour \( u, v, w \in ]m, b] \))
	\item[\(\lceil 2 \rfloor\)] soit \( u, v \in [a, m[ ~ et ~ w \in ]m, b] \) (mêmes arguments pour \( u, v, w \in [a, m[ \))
	\item[\(\lceil 3 \rfloor\)] soit u, v ou w est notre m.
\end{enumerate}
Montrons donc que tous les cas sont impossibles.
\begin{enumerate}
	\item[\(\lceil 1 \rfloor\)] on sait que \emph{f est strictement décroissante} sur [m, b] (et donc strictement décroissante sur ]m, b] par inclusion) donc \( v < w \Rightarrow f(v) > f(w) \) or ce sont 2 racines de f. \textbf{Contradiction}.
	\item[\(\lceil 2 \rfloor\)] on sait que \emph{f est strictement croissante} sur [a, m] (et donc strictement croissante sur [a, m[ par inclusion) donc \( u < v \Rightarrow f(u) < f(v) \) or ce sont 2 racines de f. \textbf{Contradiction}.
	\item[\(\lceil 3 \rfloor\)] $~$ \newline
	\begin{enumerate}
		\item[\(\lceil 3.1 \rfloor\)] Si u est m, alors \( u, v ~ et ~ w \in [m, b] \) et par \(\lceil 1 \rfloor\), c'est impossible. \textbf{Contradiction}.
		\item[\(\lceil 3.2 \rfloor\)] Si w est m, alors \( u, v ~ et ~ w \in [a, m] \) et par \(\lceil 2 \rfloor\), c'est impossible. \textbf{Contradiction}.
		\item[\(\lceil 3.3 \rfloor\)] Si v est m, alors \( \forall x \in [a, m[, ~ \forall y \in ]m, b], ~ f(x) < f(v) \) (par la stricte croissance de f sur \( [a, m] \)) et \( f(v) > f(y)\) (par la stricte décroissance de f sur \( [m, b] \)). Cependant, en particulier, \( x = u \) et \( y = v \) et qui sont racines de f. Donc, \( 0 = f(u) < f(v) = 0 \) et \( 0 = f(u) > f(w) = 0 \) ce qui est impossible. \textbf{Contradiction}.
	\end{enumerate}
\end{enumerate}

\newpage

On suppose maintenant pour le reste de la question 3 que \( f \in \mathcal{C}^{2}([a, b], \mathbb{R}) \). De plus, on suppose que toutes les racines de la dérivée de \( f \) sont simples.

\subsection{Partie B:}

On doit montrer que la dérivée de \( f \) possède une unique racine dans \( ]a, b[ \).

\subsubsection{Existence de la racine}

Montrons d'abord que la dérivée de \( f \) a une racine dans \( ]a, b[ \). Par l'hypothèse \( f \in \mathcal{C}^{2}([a, b], \mathbb{R}) \), on sait que \( \partial f \) est continue.
On sait également par unimodalité que
$$
\exists m \in ]a, b[,~ f \text{ est croissante sur } ]a, m[ \text{ et } f \text{ est décroissante sur } ]m, b[
$$
Comme la dérivée d'une fonction croissante (resp. décroissante) est positive (resp. négative), et comme \( (a+m)/2 \in ]a, m[ \) et \( (m+b)/2 \in ]m, b[ \), on sait que
$$
\partial f ((a+m)/2) \geq 0 \text{ et } \partial f((m+b)/2) \leq 0
$$
On peut alors séparer en 3 cas exhaustifs :
\begin{itemize}
    \item Cas 1 : \( f ((a+m)/2) = 0 \) \\
    Alors, comme \( (a+m)/2 \in ]a, m[ \subset ]a, b[ \), on a bien trouvé une racine de \( \partial f \) dans \( ]a, b[ \).
    \item Cas 2 : \( f ((m+b)/2) = 0 \) \\
    Alors, comme \( (m+b)/2 \in ]m, b[ \subset ]a, b[ \), on a bien trouvé une racine de \( \partial f \) dans \( ]a, b[ \).
    \item Cas 3 : \( \partial f ((a+m)/2) > 0 \) et \( \partial f((m+b)/2) < 0 \) \\
    Comme on sait que \( \partial f \) est continue, et comme \( \partial f ((a+m)/2) \partial f((m+b)/2) < 0 \), on peut appliquer le théorème des valeurs intermédiaires et on obtient
    $$
    \exists \xi \in \left] \frac{a+m}{2}, \frac{m+b}{2} \right[,~ \partial f (\xi) = 0
    $$
    Comme \( \xi \in ](a+m)/2, (m+b)/2[ \subset ]a, b[ \), on a bien trouvé une racine de \( \partial f \) dans \( ]a, b[ \).
\end{itemize}
Par exhaustivité des cas, on a bien montré l'existence d'une racine de \( \partial f \) dans \( ]a, b[ \).

\subsubsection{Unicité de la racine}

\begin{itemize}
	\item[$\bullet$] \underline{Montrons que $\partial f(m) =0 $} \\
		On sait, car $f$ est unimodale, que :
		\begin{itemize}
			\item[-] $f$ est strictement croissante sur $[a,m]$ donc pour tout $x$ dans $[a,m]$, $\partial f(x) \geqslant 0$. En particulier $\partial f(m) \geqslant 0$.
			\item[-] $f$ est strictement décroissante sur $[m,b]$ donc pour tout $x$ dans $[m,b]$, $\partial f(x) \leqslant 0 $. En particulier $\partial f(m) \leqslant 0$
		\end{itemize}
		$\Rightarrow$ On a donc $\partial f(m) = 0$
	\item[$\bullet$] \underline{Montrons l'unicité de la racine de la dérivée} \\
		On suppose sans perte de généralité l'existance d'un $\xi \in [a,m]$ tel que $ \partial f(\xi) = 0 = \partial f(m)$ avec $ m \neq \xi$. Donc $\xi$ est dans $[a,m[$. Comme $f$ est strictement croissante sur $[a,m]$ par définition de $f$, on a que pour tout $x$ dans $[a,m]$, $\partial f(x) \geqslant 0$. Donc, pour tout $x$ dans $[a,m]$, $\partial f(x) \geqslant \partial f(\xi) = 0$. Ceci implique que $\xi$ est un minimum local de $\partial f$, donc $\partial^2 f(\xi) = 0$. Or par hypothèse sur $f$, $\xi$ est une racine simple de $\partial f$ (càd $\partial^2 f(\xi) \neq 0$). On a une contradiction. Donc on a l'unicité.
\end{itemize}


\newpage

\subsection{Partie C:}

\begin{algorithm}

\begin{algorithmic}[1]

\REQUIRE \( df \), dérivée de \( f \) ; \( a, b  \in \mathbb{R} \) avec \( a < b \).
\ENSURE Le singleton contenant l'unique racine de la fonction \( df \) dans l'intervalle \( ]a, b[ \)
\STATE \( dfa \leftarrow df(a) \)
\STATE \( dfb \leftarrow df(b) \)
\STATE \( prec \leftarrow 10^{-10} \)
\IF {\(dfa > 0 ~ et ~ dfb < 0 \)}
	\STATE \( x \leftarrow Root1D.brent(df, a, b, tol=prec) \)
	\RETURN $x$
\ELSE
	\STATE \( mid \leftarrow \frac{a + b}{2} \)
	\STATE \( dfm \leftarrow df(mid) \)
	\IF {\( dfm = 0 \)}
		\RETURN $mid$
	\ELSIF {\( dfm > 0 \)}
		\RETURN \( rootDeriv(df, mid, b) \)
	\ELSE
		\RETURN \( rootDeriv(df, a, mid) \)
	\ENDIF
\ENDIF
\caption{\textbf{Algorithme : }rootDeriv}
\end{algorithmic}

\end{algorithm}%---------------------------------------------------------------------------
\begin{algorithm}

\begin{algorithmic}[1]

\REQUIRE \( f \), une fonction unimodale ; \( df \), dérivée de \( f \) ; \( a, b  \in \mathbb{R} \) avec \( a < b \).
\ENSURE L'ensemble des solutions de \( f \) dans l'intervalle \( [a, b] \).
\STATE \( fa \leftarrow f(a) \)
\STATE \( fb \leftarrow f(b) \)
\IF {\( fa > 0 ~ et ~ fb > 0 \)}
	\RETURN  $[]$
\ELSIF {\( fa = 0 ~ et ~ fb > 0 \)}
	\RETURN $[a]$
\ELSIF {\( fa > 0 ~ et ~ fb = 0 \)}
	\RETURN $[b]$
\ELSE
	\STATE \( m \leftarrow rootDeriv(df, a, b) \)
	\STATE \( prec \leftarrow 10^{-10}\)
	\IF {\( fa > 0 ~ et ~ fb < 0 \)}
		\STATE \( x \leftarrow Root1D.brent(f, m, b, tol=prec) \)
		\RETURN $[x]$
	\ELSIF {\( fa < 0 ~ et ~ fb > 0 \)}
		\STATE \( x \leftarrow Root1D.brent(f, a, m, tol=prec) \)
		\RETURN $[x]$
	\ELSE
		\STATE \( fm \leftarrow f(m) \)
		\IF {\( fm > 0 \)}
			\STATE \( x \leftarrow Root1D.brent(f, a, m, tol=prec) \)
			\STATE \( y \leftarrow Root1D.brent(f, m, b, tol=prec) \)
			\RETURN $[x, y]$
		\ELSIF {\( fm = 0 \)}
			\RETURN $[m]$
		\ELSE
			\RETURN $[]$
		\ENDIF
	\ENDIF
\ENDIF
\caption{\textbf{Algorithme : }rootFinding}
\end{algorithmic}

\end{algorithm}

\newpage

\subsection{Partie D:}

Il faut maintenant prouver que nos algorithmes sont corrects.

\subsubsection{Preuve de rootDeriv}

Cette fonction prend en entrée la dérivée de \( f \), notée ici \( df \), sachant que \( f \) est une fonction unimodale de classe \( \mathcal{C}^2([a, b], \mathbb{R}) \), ainsi que \( a, b \in \mathbb{R} \) tels que \( a < b \). \\
On sait également par hypothèse que \( df \) ne possède que des zéros simples.
On doit montrer que la fonction retourne bien l'unique racine de \( df \) dans l'intervalle \( ]a, b[ \). \\
Tout d'abord, on a montré précédemment (Exercice 3.B) que \( df \) a bien une unique racine dans l'intervalle \( ]a, b[ \). Il reste à montrer que le programme retourne bien cette racine. \\
Pour cela, séparons les entrées en cas exhaustifs.
\begin{itemize}

    \item Cas 1 : \( df(a) > 0 \) et \( df(b) < 0 \) \\
Dans ce cas, on sait que \( df(a) \cdot df(b) < 0 \) et on peut donc simplement appeler la fonction Root1D.brent qui renvoie bien l'unique racine de \( df \).

    \item Cas 2 : \( df(a) = 0 \) ou \( df(b) = 0 \) \\
On calcule alors \( mid \), la moyenne entre \( a \) et \( b \). On sait donc que \( mid \in ]a, b[ \). On sépare alors en 3 cas selon le signe de \( df(mid) \).
    \begin{itemize}

        \item Cas 2.1 : \( df(mid) = 0 \) \\
Alors, comme \( mid \in ]a, b[ \), il s'agit bien de la racine qu'on cherche.

        \item Cas 2.2 : \( df(mid) > 0 \) \\
On sait que
$$
(\forall x \in [a, m],~ df(x) \geq 0) \land (\forall x \in [m, b],~ df(x) \leq 0)
$$
et donc
$$
\forall x \in ]a, b[,~ (df(x) > 0 \Longrightarrow x \in ]a, m[) \land (df(x) < 0 \Longrightarrow x \in ]m, b[)
$$
Comme \( mid \in ]a, b[ \) et \( df(mid) > 0 \), on en déduit que \( mid \in ]a, m[ \) ou encore \( m \in ]mid, b[ \). On relance donc la fonction rootDeriv avec cette fois comme paramètres \( df,~ mid \) et \( b \).

    \item Cas 2.3 : \( df(mid) < 0 \) \\
Par des justifications analogues à celles du cas 2.2, on sait que \( m \in ]a, mid[ \) et on relance donc la fonction rootDeriv avec cette fois comme paramètres \( df,~ a \) et \( mid \).

    \end{itemize}

\end{itemize}
Puisque \( f \) est unimodale, on sait en particulier que \( f \) est croissante sur \( [a, m[ \) et décroissante sur \( ]m, b] \) et donc que \( df(a) \geq 0 \) et \( df(b) \leq 0 \). Les cas 1 et 2 sont donc exhaustifs. \\
Le cas 1 s'arrête évidemment car il respecte les entrées de la fonction Root1D.brent et donne bien la racine de \( df \) dans \( ]a, b[ \). \\
La cas 2.1 retourne la bonne valeur comme expliqué plus haut. \\
En ce qui concerne le cas 2, si on considère qu'on ne tombe jamais dans le cas 2.1, on voit qu'après une itération, au moins une de nos bornes est non nulle car la distance entre nos bornes a été divisée par 2 donc au moins une d'entre elles appartient à l'ensemble \( ]a, b[ \) avec les \( a \) et \( b \) initiaux. \\
Si la deuxième borne ne devenait jamais nullle, puisque la distance entre les deux bornes tend vers 0, et comme \( m \) est toujours entre les deux bornes, on aurait alors par passage à la limite que \( m = a \) ou \( m = b \) ce qui est impossible vu que par hypothèse \( m \in ]a, b[ \) avec \( a \neq b \).
Donc les deux bornes deviendront toutes deux non nulles en un temps fini et on entrera dans le cas 1.

\newpage

\subsubsection{Preuve de rootFinding}

Cette fonction prend 4 entrées :
\begin{itemize}
    \item Une fonction \( f \) unimodale de classe \( \mathcal{C}^2([a, b], \mathbb{R}) \)
    \item La dérivée de \( f \), notée \( df \). Elle ne doit avoir que des zéros simples.
    \item Un réel \( a \), la borne gauche de l'intervalle sur lequel on cherche les solutions
    \item Un réel \( b \), la borne droite de l'intervalle sur lequel on cherche les solutions
\end{itemize}
Il faut montrer que ce programme retourne bien toutes les racines de \( f \). \\
Pour cela, séparons en cas exhaustifs.
\begin{itemize}

    \item Cas 1 : \( f(a) > 0 \) et \( f(b) > 0 \) \quad (on rentre dans la condition de la ligne 3) \\
Comme \( f \) est croissante sur \( [a, m] \), on sait que \( \forall x \in [a, m],~ f(x) \geq f(a) > 0 \). \\
De même, comme \( f \) est décroissante sur \( [m, b] \), on sait que \( \forall x \in [m, b],~ f(x) \geq f(b) > 0 \). \\
On a alors \( \forall x \in [a, b],~ f(x) > 0 \) et la fonction n'a donc pas de racines sur \( [a, b] \). On retourne donc la liste vide.

    \item Cas 2 : \( f(a) = 0 \) et \( f(b) > 0 \) \quad (on rentre dans la condition de la ligne 5) \\
On sait alors que \( a \) est une racine de \( f \). \\
Par stricte croissance de \( f \) sur \( [a, m] \), on sait que \( \forall x \in ]a, m],~ f(x) > f(a) = 0 \). \\
De même, par décroissance de \( f \) sur \( [m, b] \), on sait que \( \forall x \in [m, b],~ f(x) \geq f(b) > 0 \). \\
On a alors \( \forall x \in ]a, b],~ f(x) > 0 \) et la fonction n'a donc pas de racines sur \( ]a, b] \). On retourne donc le singleton \( \{a\} \), \( a \) étant la seule racine de \( f \) dans \( [a, b] \).

    \item Cas 3 : \( f(a) > 0 \) et \( f(b) = 0 \) \quad (on rentre dans la condition de la ligne 7) \\
De manière analogue au cas 2, par croissance de \( f \) sur \( [a, m] \) et par stricte décroissance de \( f \) sur \( [m, b] \), on voit que \( b \) est l'unique racine de \( f \) et on retourne donc le singleton \( \{b\} \).

    \item Cas 4 : \( f(a) < 0 \) ou \( f(b) < 0 \) ou \( f(a) = f(b) = 0 \) \quad (on rentre dans la condition de la ligne 9) \\
On sait par l'exercice 3.B que \( m \) est l'unique racine de \( df \) dans l'intervalle \( ]a, b[ \). On utilise donc la fonction rootDeriv pour calculer la valeur de \( m \).

    \begin{itemize}
    
        \item Cas 4.1 : \( f(a) > 0 \) et \( f(b) < 0 \) \quad (on rentre dans la condition de la ligne 12) \\
Par croissance de \( f \) sur \( [a, m] \), on sait que \( \forall x \in [a, m],~ f(x) \geq f(a) > 0 \). Il n'y a donc pas de racines dans \( [a, m] \) \\
Comme \( f(m) > 0 \) par l'argument ci-dessus et \( f(b) < 0 \) par hypothèse, on a \( f(m) \cdot f(b) < 0 \) et donc Root1D.brent nous donne une racine de \( [m, b] \). Par stricte décroissance de \( f \) sur \( [m, b] \), cette racine est unique et on renvoie donc bien toutes les racines de \( f \) sur \( [a, b] \).

        \item Cas 4.2 : \( f(a) < 0 \) et \( f(b) > 0 \) \quad (on rentre dans la condition de la ligne 15) \\
Par des arguments analogues au cas 4.1, en appliquant Root1D.brent sur l'intervalle \( [a, m] \), on trouve l'unique racine de \( f \) dans l'intervalle \( [a, b] \).

        \item Cas 4.3 : \( f(a) \leq 0 \) et \( f(b) \leq 0 \) \quad (on rentre dans la condition de la ligne 18) \\
On divise encore en cas selon le signe de \( m \).

        \begin{itemize}
        
            \item Cas 4.3.1 : \( f(m) > 0 \) \quad (on rentre dans la condition de la ligne 20) \\
On sait que \( f \) est strictement croissante sur \( [a, m[ \) et strictement décroissante sur \( ]m, b] \) et \( f(a) \leq 0 \) et \( f(b) \leq 0 \) donc il y a une unique racine dans chacun de ces deux intervalles. Comme \( f(a) \cdot f(m) \leq 0 \) et \( f(m) \cdot f(b) \leq 0 \), on trouve bien ces deux racines via Root1D.brent avec les bornes \( a \) et \( m \) et encore une fois Root1D.brent avec les bornes \( m \) et \( b \).
            
            \item Cas 4.3.2 : \( f(m) = 0 \) \quad (on rentre dans la condition de la ligne 24) \\
Par stricte croissance de \( f \) sur \( [a, m] \) et stricte décroissance de \( f \) sur \( [m, b] \), on sait que \( \forall x \in [a, m[ \cup ]m, b],~ f(x) < f(m) = 0 \). \\
On sait que donc \( m \) est l'unique racine de \( f \) sur \( [a, b] \).
            
            \item Cas 4.3.3 : \( f(m) < 0 \) \quad (on rentre dans la condition de la ligne 26) \\
Par stricte croissance de \( f \) sur \( [a, m] \) et stricte décroissance de \( f \) sur \( [m, b] \), comme \( f(m) < 0 \), il n'y a aucune racine de \( f \) dans l'intervalle \( [a, b] \), on retourne donc la liste vide.
            
        \end{itemize}
    
    \end{itemize}

\end{itemize}
Par exhaustivité de nos cas, on a bien prouvé que notre algorithme fonctionne.

\newpage

\section{Exercice 6}

Cette fonction prend deux réels \( a \) et \( y \). On doit montrer que ce programme retourne bien l'ensemble des racines de la fonction \( f : x \mapsto x + a \sin(x) - y \). \\
Séparons pour cela en cas exhaustifs.
\begin{itemize}

    \item Cas 1 : \( |a| \leq 1 \) \\
Commençons par montrer que \( f \) est strictement croissante. Començons par remarquer que \( \partial f : x \mapsto 1 + a \cos(x) \). Si \( a = 0 \), \( \partial f \) est strictement positive car de constante 1 et \( f \) est donc strictement croissante. Sinon, on sait que
$$
1 + a \cos(x) = 0 \Longleftrightarrow \cos(x) = \frac{-1}{a}
$$
Or, comme \( |a| \leq 1 \), on a \( \left|\frac{-1}{a}\right| \geq 1 \) ce qui nous donne 3 nouveaux cas :

    \begin{itemize}
    
        \item Cas 1.1 : \( |a| < 1 \land a \neq 0 \) \\
Alors, on a \( \left|\frac{-1}{a}\right| > 1 \) et donc \( \partial f \) n'a aucune racine. On remarque que \( \partial f (0) = 1 + a \cos(0) = 1 + a > 0 \) car \( |a| < 1 \). Donc, par continuité de \( \partial f \) sur \( \mathbb{R} \), on a que \( \partial f \) est strictement positive sur \( \mathbb{R} \) et donc \( f \) est strictement croissante sur \( \mathbb{R} \). \\

        \item Cas 1.2 : \( a = 1 \) \\
Alors, on a \( \partial f(x) = 0 \Longleftrightarrow x = 2k\pi + \pi \) avec \( k \in \mathbb{Z} \). \\
On voit que \( \forall k \in \mathbb{Z},~ 2k\pi + 2\pi \in ]2k\pi + \pi, 2(k+1)\pi + \pi[ \) et \( \partial f(2k\pi + 2\pi) = 2 > 0 \). \\
Donc, par continuité de \( \partial f \) sur \( \mathbb{R} \), on a que \( \partial f \) est strictement positive sur \\
\( \displaystyle \bigcup_{k \in \mathbb{Z}} ]2k\pi + \pi, 2(k+1)\pi + \pi[ \) et donc \( f \) est strictement croissante sur \\
\( \displaystyle \bigcup_{k \in \mathbb{Z}} [2k\pi + \pi, 2(k+1)\pi + \pi] = \mathbb{R} \).

        \item Cas 1.3 : \( a = -1 \) \\
Par des arguments analogues au cas 1.2, on voit que \( \partial f \) est strictement positive sur \( \displaystyle \bigcup_{k \in \mathbb{Z}} ]2k\pi, 2(k+1)\pi[ \) et donc \( f \) est strictement croissante sur \( \displaystyle \bigcup_{k \in \mathbb{Z}} [2k\pi, 2(k+1)\pi] = \mathbb{R} \).
        
    \end{itemize}
On remarque que
\begin{align*}
    |\sin(y-1)| \leq 1 & \Longleftrightarrow |a\sin(y-1)| \leq 1 \text{ car } |a| \leq 1 \\
    & \Longrightarrow a\sin(y-1) \leq 1 \\
    & \Longrightarrow f(y-1) = y - 1 + a \sin(y-1) - y = a\sin(y-1) - 1 \leq 0
\end{align*}
On voit aussi que
\begin{align*}
    |\sin(y+1)| \leq 1 & \Longleftrightarrow |a\sin(y-1)| \leq 1 \text{ car } |a| \leq 1 \\
    & \Longrightarrow -1 \leq a\sin(y+1) \\
    & \Longrightarrow f(y+1) = y + 1 + a \sin(y+1) - y = a\sin(y+1) + 1 \geq 0
\end{align*}
Comme \( f \) est strictement croissante sur \( \mathbb{R} \), elle a une unique racine dans \( \mathbb{R} \) et plus précisément dans l'intervalle \( [y-1, y+1] \). Comme \( f(y-1) \neq f(y+1) \) par stricte croissance, et comme \( f(y-1) \cdot f(y+1) \leq 0 \), on peut calculer cette unique racine en appliquant Root1D.brent avec \( y-1 \) et \( y+1 \) comme bornes.
    
    \item Cas 2 : \( a > 1 \) \\
On sait que
\begin{align*}
    \partial f(x) = 0 & \Longleftrightarrow 1 + a \cos(x) = 0 \\
    & \Longleftrightarrow \cos(x) = \frac{-1}{a} && \text{car } a \neq 0 \\
    & \Longleftrightarrow x = 2k \pi \pm \arccos\left(\frac{-1}{a}\right) && \text{avec } k \in \mathbb{Z} \text{ car } |a| > 1 \text{ donc } \left|\frac{-1}{a}\right| < 1
\end{align*}
Cherchons sur quels intervalles \( \partial f \) est positive ou négative. On remarque que
$$
\forall k \in \mathbb{Z},~ 2k \pi - \arccos\left(\frac{-1}{a}\right) < 2k \pi + \arccos\left(\frac{-1}{a}\right) < 2(k+1) \pi - \arccos\left(\frac{-1}{a}\right)
$$
Ces bornes étant toutes les racines de \( \partial f \), il suffit de trouver le signe d'un point entre chacune de ces bornes pour connaître le signe de \( \partial f \) entre chaque borne. \\
Sur les intervalles \( \left[ 2k \pi - \arccos\left(\frac{-1}{a}\right),~ 2k \pi + \arccos\left(\frac{-1}{a}\right) \right] \), \( \partial f \) est positive car
$$
\forall k \in \mathbb{Z},~ 2k \pi \in \left[ 2k \pi - \arccos\left(\frac{-1}{a}\right),~ 2k \pi + \arccos\left(\frac{-1}{a}\right) \right] \text{ et } \partial f(2k \pi) = 1 + a > 0
$$
Sur les intervalles \( \left[ 2k \pi + \arccos\left(\frac{-1}{a}\right),~ 2(k+1) \pi - \arccos\left(\frac{-1}{a}\right) \right] \), \( \partial f \) est négative car
$$
\forall k \in \mathbb{Z},~ 2k \pi + \pi \in \left[ 2k \pi + \arccos\left(\frac{-1}{a}\right),~ 2(k+1) \pi - \arccos\left(\frac{-1}{a}\right) \right] \text{ et } \partial f(2k \pi + \pi) = 1 - a < 0
$$
Donc, par continuité de \( \partial f \) sur \( \mathbb{R} \), \( \forall k \in \mathbb{Z} \)
$$
\partial f \text{ est strictement positive sur } \left] 2k \pi - \arccos\left(\frac{-1}{a}\right),~ 2k \pi + \arccos\left(\frac{-1}{a}\right) \right[
$$
et
$$
\partial f \text{ est strictement négative sur } \left] 2k \pi + \arccos\left(\frac{-1}{a}\right),~ 2(k+1) \pi - \arccos\left(\frac{-1}{a}\right) \right[
$$
et donc
$$
f \text{ est strictement croissante sur } \left[ 2k \pi - \arccos\left(\frac{-1}{a}\right),~ 2k \pi + \arccos\left(\frac{-1}{a}\right) \right]
$$
et
$$
f \text{ est strictement décroissante sur } \left[ 2k \pi + \arccos\left(\frac{-1}{a}\right),~ 2(k+1) \pi - \arccos\left(\frac{-1}{a}\right) \right]
$$
ou encore
$$
f \text{ est unimodale sur } \left[ 2k \pi - \arccos\left(\frac{-1}{a}\right),~ 2(k+1) \pi - \arccos\left(\frac{-1}{a}\right) \right]
$$
En appliquant notre fonction rootFinding contruite à l'exercice 3 sur ces intervalles, on trouvera donc toutes les racines de \( f \).

    \item Cas 3 : \( a < -1 \) \\
Par des arguments analogues au cas 2,
$$
f \text{ est unimodale sur } \left[ 2k \pi + \arccos\left(\frac{-1}{a}\right),~ 2(k+1) \pi + \arccos\left(\frac{-1}{a}\right) \right]
$$
En appliquant notre fonction rootFinding contruite à l'exercice 3 sur ces intervalles, on trouvera donc toutes les racines de \( f \).
\end{itemize}
On a donc bien montré que notre algorithme retourne toutes les racines de \( f \).

\end{document}
